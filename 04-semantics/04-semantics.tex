% (The MIT License)
%
% Copyright (c) 2022-2025 Yegor Bugayenko
%
% Permission is hereby granted, free of charge, to any person obtaining a copy
% of this software and associated documentation files (the 'Software'), to deal
% in the Software without restriction, including without limitation the rights
% to use, copy, modify, merge, publish, distribute, sublicense, and/or sell
% copies of the Software, and to permit persons to whom the Software is
% furnished to do so, subject to the following conditions:
%
% The above copyright notice and this permission notice shall be included in all
% copies or substantial portions of the Software.
%
% THE SOFTWARE IS PROVIDED 'AS IS', WITHOUT WARRANTY OF ANY KIND, EXPRESS OR
% IMPLIED, INCLUDING BUT NOT LIMITED TO THE WARRANTIES OF MERCHANTABILITY,
% FITNESS FOR A PARTICULAR PURPOSE AND NONINFRINGEMENT. IN NO EVENT SHALL THE
% AUTHORS OR COPYRIGHT HOLDERS BE LIABLE FOR ANY CLAIM, DAMAGES OR OTHER
% LIABILITY, WHETHER IN AN ACTION OF CONTRACT, TORT OR OTHERWISE, ARISING FROM,
% OUT OF OR IN CONNECTION WITH THE SOFTWARE OR THE USE OR OTHER DEALINGS IN THE
% SOFTWARE.

\documentclass{article}
\usepackage{../ppa}
\newcommand*\thetitle{Formal Semantics}
\newcommand*\thesubtitle{Rules, Structural, Natural, Reduction}
\begin{document}

\lnTitlePage{4}{10}{xtu4WODOU1M}

\pptToc

While syntax is a representation of a program, semantics \(S(P)\)
is a formal description of execution of \(P\): reachable states, execution traces, etc.

\plush{\pptChapter[Rules]{Rules, Axioms, Trees}}

\pptSection[Inference]{Inference Rule}

A \emph{proof system} is formed from a set of \emph{inference rules} chained together to form proofs, also called \emph{derivations}. Any derivation has only one final conclusion, which is the statement proved or derived. (Wiki)

\begin{prooftree}
\AxiomC{$\vdash a < b$}
\AxiomC{$\vdash b < c$}
\RightLabel{R1}
\BinaryInfC{$a < c$}
\end{prooftree}

Premises (known \emph{facts}): $a < b$ and $b < c$. \\
{\small (antecedent)}

Conclusion (new fact): $a < c$. \\
{\small (consequent)}

\plush{}

\pptSection{Axiom}

An \emph{axiom} is an inference rule without a premise.

\begin{prooftree}
\AxiomC{}
\RightLabel{$A_1$}
\UnaryInfC{$\vdash x \times 0 = 0 $}
\end{prooftree}

It reads: in any environment, the result of multiplication of $x$ by zero equals to zero.

\plush{}

\pptSection[Transition]{Transition Rule}

A \emph{transition rule} defines the conditions under which a system may be moved to a new \emph{state}.
\begin{prooftree}
\AxiomC{$ \langle \texttt{a}, s \rangle \longrightarrow \langle n, s\rangle $}
\UnaryInfC{$ \langle \texttt{a++}, s \rangle \longrightarrow \langle n, s[\texttt{a} \mapsto n + 1]\rangle $}
\end{prooftree}
It reads: if \texttt{a} produces $n$ without changing the state, then \texttt{a++} may produce $n$ changing the state by adding a new mapping $\texttt{a}\mapsto n$.

\plush{}

The following set of transition rules may constitute the entire semantic of a language:
{\small\begin{prooftree}
\AxiomC{\strut}
\UnaryInfC{$ \langle \texttt{x}, s \rangle \longrightarrow \langle s[\texttt{x}], s\rangle $}
\DisplayProof
\quad\quad
\AxiomC{\strut}
\UnaryInfC{$ \langle n, s \rangle \longrightarrow \langle n, s\rangle $}
\DisplayProof

\AxiomC{$ \langle \texttt{y}, s \rangle \longrightarrow \langle n, s\rangle $}
\UnaryInfC{$ \langle \texttt{x:=y}, s \rangle \longrightarrow \langle \texttt{skip}, s[\texttt{x} \mapsto n]\rangle $}
\DisplayProof

\AxiomC{$ \langle C_1, s \rangle \longrightarrow \langle \texttt{skip}, s'\rangle $}
\AxiomC{$ \langle C_2, s' \rangle \longrightarrow \langle n, s''\rangle $}
\BinaryInfC{$ \langle C_1;C_2, s \rangle \longrightarrow \langle n, s''\rangle $}
\DisplayProof

\AxiomC{$ \langle \texttt{x}, s \rangle \longrightarrow \langle n, s\rangle $}
\UnaryInfC{$ \langle \texttt{x++}, s \rangle \longrightarrow \langle n, s[\texttt{x} \mapsto n + 1]\rangle $}
\end{prooftree}}

\plush{}

\pptSection[Tree]{Proof Tree}

Let's prove that \texttt{a:=5;a++++;} equals to \texttt{6}:
\begin{pptWide}{2}
Transition rules:
{\scriptsize\begin{prooftree}
\def\extraVskip{3pt}
\AxiomC{\strut}
\UnaryInfC{$ \langle \texttt{x}, s \rangle \longrightarrow \langle s[\texttt{x}], s\rangle $}
\DisplayProof
\quad\quad
\AxiomC{\strut}
\UnaryInfC{$ \langle n, s \rangle \longrightarrow \langle n, s\rangle $}
\DisplayProof

\AxiomC{$ \langle \texttt{y}, s \rangle \longrightarrow \langle n, s\rangle $}
\UnaryInfC{$ \langle \texttt{x:=y}, s \rangle \longrightarrow \langle \texttt{skip}, s[\texttt{x} \mapsto n]\rangle $}
\DisplayProof

\AxiomC{$ \langle C_1, s \rangle \longrightarrow \langle \texttt{skip}, s'\rangle $}
\AxiomC{$ \langle C_2, s' \rangle \longrightarrow \langle n, s''\rangle $}
\BinaryInfC{$ \langle C_1;C_2, s \rangle \longrightarrow \langle n, s''\rangle $}
\DisplayProof

\AxiomC{$ \langle \texttt{x}, s \rangle \longrightarrow \langle n, s\rangle $}
\UnaryInfC{$ \langle \texttt{x++}, s \rangle \longrightarrow \langle n, s[\texttt{x} \mapsto n + 1]\rangle $}
\end{prooftree}}
\par\columnbreak\par
Proof tree:
{\scriptsize\begin{prooftree}
\AxiomC{}
\UnaryInfC{$ \langle 5, \{\} \rangle \longrightarrow \langle 5, \{ \}\rangle $}
\UnaryInfC{$ \langle \texttt{a:=5}, \{\} \rangle \longrightarrow \langle \texttt{skip}, \{ \texttt{a} \mapsto 5 \}\rangle $}
\AxiomC{}
\UnaryInfC{$ \langle \texttt{a}, \{ \texttt{a} \mapsto 5 \} \rangle \longrightarrow \langle 5, \{ \texttt{a} \mapsto 5 \}\rangle$}
\UnaryInfC{$ \langle \texttt{a++}, \{ \texttt{a} \mapsto 5 \} \rangle \longrightarrow \langle 5, \{ \texttt{a} \mapsto 6 \}\rangle $}
\UnaryInfC{$ \langle \texttt{a++++}, \{ \texttt{a} \mapsto 5 \} \rangle \longrightarrow \langle 6, \{ \texttt{a} \mapsto 7 \}\rangle $}
\BinaryInfC{$ \langle \texttt{a:=5;a++++}, \{ \} \rangle \longrightarrow \langle 6, \{ \texttt{a} \mapsto 7 \}\rangle $}
\end{prooftree}}
\end{pptWide}

\plush{}

\plush{\pptChapter[Vs.]{Operational vs Denotational}}

The \emph{denotational semantics} assign to every expression the \emph{number} denoted by that expression:
\begin{gather*}
\mathrel{\bm{\Downarrow}} \subseteq \mathcal{A} \times \mathcal{D} \\
x^n \bm{\Downarrow} y \quad\text{where}\; y = x \times x \times x \dots \times x \quad\text{($n$ times)}
\end{gather*}

The \emph{operational semantics} describe the computation steps taken in order to evaluate the expression to \emph{normal form}:
\begin{gather*}
\mathrel{\bm{\leadsto}} \subseteq \mathcal{A} \times \mathcal{A} \\
\text{1)}\; x^n \bm{\leadsto} x \times x^{n-1} \quad\text{if}\; x > 0 \quad\quad \text{2)}\; x^0 \bm{\leadsto} 1
\end{gather*}

The operational semantics is the specification of an \emph{interpreter} for the programming language whereas the denotational semantics tries to capture the \emph{mathematical essence} of a program, abstracting away from computational details.

\plush{}

\plush{\pptChapter[Natural]{Natural Semantic (Denotational)}}

Syntax: {\ttfamily FASTER; STOP; SLOWER;}\\
Semantic ($\quad\bm{\Downarrow} \subseteq \langle \texttt{COMMAND}, \mathbb{N} \rangle \times \langle \mathbb{B}, \mathbb{N} \rangle\quad$):
\begin{gather*}
\frac
  {}
  {\langle \texttt{STOP}, s \rangle \Downarrow \langle \textbf{tt}, 0 \rangle }
  \text{R1}
\\[12pt]
\frac
  {\textcolor{gray}{ s < 60 }}
  {\langle \texttt{FASTER}, s \rangle \Downarrow \langle \textbf{tt}, s + 20 \rangle }
  \text{R2}
\quad
\frac
  {\textcolor{gray}{  s \geq 60 }}
  {\langle \texttt{FASTER}, s \rangle \Downarrow \langle \textbf{ff}, s \rangle }
  \text{R3}
\\[12pt]
\frac
  {}
  {\langle \texttt{SLOWER}, s \rangle \Downarrow \langle \textbf{tt}, \text{max}(0, s - 20) \rangle }
  \text{R4}
\\[12pt]
\frac
  {\langle C_1, s \rangle \Downarrow \langle r_1, s' \rangle \quad \langle C_2, s' \rangle \Downarrow \langle r_2, s'' \rangle }
  {\langle C_1 ; C_2, s \rangle \Downarrow \langle r_1 \wedge r_2, s'' \rangle }
  \text{R5}
\end{gather*}
Introduced by Gilles Kahn in 1987.
\plush{}

\pptSection[Tree]{Proof Tree}

\begin{prooftree}
\AxiomC{\textcolor{gray}{$45 < 60$}}
\RightLabel{R2}
\UnaryInfC{$ \langle \texttt{FASTER}, 45 \rangle \Downarrow \langle \textbf{tt}, 65 \rangle $}
\AxiomC{\textcolor{gray}{$65 > 60$}}
\RightLabel{R3}
\UnaryInfC{$ \langle \texttt{FASTER}, 65 \rangle \Downarrow \langle \textbf{ff}, 65 \rangle $}
\RightLabel{R5}
\BinaryInfC{$ \langle \texttt{FASTER}; \texttt{FASTER}, 45 \rangle \Downarrow \langle \textbf{ff}, 65 \rangle $}
\AxiomC{$ \langle \texttt{SLOWER}, 65 \rangle \Downarrow \langle \textbf{tt}, 45 \rangle $}
\RightLabel{R5}
\BinaryInfC{$ \langle \texttt{FASTER}; \texttt{FASTER}; \texttt{SLOWER}, 45 \rangle \Downarrow \langle \textbf{ff}, 45 \rangle $}
\end{prooftree}

The tree is built from the bottom to the top, using the rules introduced before.
The gray conditions at the top are not parts of the rules, that's why in gray.
\plush{}

\plush{\pptChapter[SOS]{Structural Semantic (Operational)}}

Consider a program:
\begin{ffcode}
x := x + 1;
\end{ffcode}
The meaning of it may be explained by the SOS rule:
\begin{equation*}
\frac
  {\langle e, s \rangle \longrightarrow \langle n, s \rangle }
  {\langle a\;\texttt{:=}\;e, s \rangle \longrightarrow \langle \texttt{skip}, s[a \mapsto n] \rangle }
\end{equation*}

It reads: If \(e\) may be \emph{evaluated} to \(n\),
then \(a\;\texttt{:=}\;e\) inserts a new binding \(a \mapsto n\) to the state,
and skips any further processing. To understand the meaning of \texttt{x+1} a new SOS
rule is required.

Introduced by Gordon Plotkin in 1981.

\plush{}

\plush{\pptChapter[Reduction]{Reduction Semantic}}

Consider a \emph{$\lambda$-expression}:
\begin{equation*}
(\lambda a.a) b
\end{equation*}
In Java it would look like this:
\begin{ffcode}
int f(int a) { return a; }
int x = f(b);
\end{ffcode}

The expression may be reduced using so called \emph{$\beta$-reduction}:
\begin{equation*}
(\lambda x.t) s \longrightarrow t[x:=s]
\end{equation*}
Thus
\begin{equation*}
(\lambda a.a) b \quad\longrightarrow\quad b
\end{equation*}

\plush{}

\pptSection[NF]{Normal Form}

A \emph{normal form} is a form that has no more possible applications of reductions.
This not a normal form:
\begin{equation*}
(\lambda a.a) ((\lambda b.b) ((\lambda c.c) d))
\end{equation*}

It may be reduced to a normal form:
\begin{equation*}
\begin{split}
\longrightarrow_\beta\quad & (\lambda a.a) ((\lambda b.b) d) \\
\longrightarrow_\beta\quad & (\lambda a.a) d \\
\longrightarrow_\beta\quad & d \\
\end{split}
\end{equation*}

No further reductions are possible.
\plush{}

\plush{\pptChapter[Literature]{Further Reading/Watching}}

Christopher Strachey (2000),\\
\emph{Fundamental Concepts in Programming Languages}

Alexander Kurz (2022),\\
\href{https://hackmd.io/@alexhkurz/Hkf6BTL6P}{Operational and Denotational Semantics}

Michael Pradel (2021),\\
\href{https://www.youtube.com/watch?v=jsBHd3-04oA}{Lectures on ``Operational Semantics''}

Gordon Plotkin (1981),\\
\href{https://web.eecs.umich.edu/~weimerw/2006-615/reading/plotkin81structural.pdf}{A Structural Approach to Operational Semantics}

\end{document}
