% (The MIT License)
%
% Copyright (c) 2022-2023 Yegor Bugayenko
%
% Permission is hereby granted, free of charge, to any person obtaining a copy
% of this software and associated documentation files (the 'Software'), to deal
% in the Software without restriction, including without limitation the rights
% to use, copy, modify, merge, publish, distribute, sublicense, and/or sell
% copies of the Software, and to permit persons to whom the Software is
% furnished to do so, subject to the following conditions:
%
% The above copyright notice and this permission notice shall be included in all
% copies or substantial portions of the Software.
%
% THE SOFTWARE IS PROVIDED 'AS IS', WITHOUT WARRANTY OF ANY KIND, EXPRESS OR
% IMPLIED, INCLUDING BUT NOT LIMITED TO THE WARRANTIES OF MERCHANTABILITY,
% FITNESS FOR A PARTICULAR PURPOSE AND NONINFRINGEMENT. IN NO EVENT SHALL THE
% AUTHORS OR COPYRIGHT HOLDERS BE LIABLE FOR ANY CLAIM, DAMAGES OR OTHER
% LIABILITY, WHETHER IN AN ACTION OF CONTRACT, TORT OR OTHERWISE, ARISING FROM,
% OUT OF OR IN CONNECTION WITH THE SOFTWARE OR THE USE OR OTHER DEALINGS IN THE
% SOFTWARE.

\documentclass[sigsoft,12pt]{acmart}
\usepackage[T1]{fontenc}
\usepackage[utf8]{inputenc}
\usepackage{lmodern}
\usepackage{bm}
\usepackage{multicol}
\usepackage{mathtools}
\usepackage{ffcode}
\usepackage{soul}
\renewcommand\emph[1]{\ul{#1}}
\begin{document}

\section{Formal Grammar}

\subsection{Notation}

Assume, we want to create a new programming language (very similar to Basic),
which will allow us to write programs that look like this:

\begin{ffcode}
10 PRINT "What is your name?"
20 INPUT X
30 PRINT "Hello,", X
\end{ffcode}

A \emph{grammar} $G$, according to Noam Chomsky (1950s), is a tuple $\langle N, T, P, S\rangle$, where:
\begin{itemize}
  \item $N = \{ P_\texttt{rogram}, I_\texttt{nteger},
    N_\texttt{ame}, T_\texttt{ext}, L_\texttt{ine}, \dots \}$ (\emph{non-terminals} or \emph{variables})
  \item $T = \{ \texttt{10}, \texttt{20},
    \texttt{PRINT}, \texttt{X}, \texttt{,}, \texttt{"Hello"}, \dots \}$ (\emph{terminals} or \emph{alphabet})
  \item $P = \{ \dots \}$ (\emph{production rules})
  \item $S \in N$ (\emph{start symbol})
\end{itemize}

Sometimes it may be $\langle V, \Sigma, R, S\rangle$, but it's not important.

A \emph{language} that can be built by $G$ is denoted as $\bm{L}(G)$.

A \emph{production rule} specifies a replacement of its \emph{left-hand side} with its \emph{right-hand side}, for example:
\begin{equation*}
\begin{split}
P_\texttt{rogram} &\to P_\texttt{rogram} \; L_\texttt{ine} \\
P_\texttt{rogram} &\to \epsilon \\
L_\texttt{ine} &\to I_\texttt{nteger} \; C_\texttt{ommand} \; T_\texttt{ail} \\
T_\texttt{ail} &\to T_\texttt{ail} \; A_\texttt{rgument} \\
T_\texttt{ail} &\to \epsilon \\
I_\texttt{nteger} &\to \texttt{10} \\
I_\texttt{nteger} &\to \texttt{20} \\
I_\texttt{nteger} &\to \dots \\
\end{split}
\end{equation*}

Formally, a production rule is (using \emph{Kleene star}):
\begin{equation*}
(T \cup N)^* N (T \cup N)^* \to (T \cup N)^*
\end{equation*}

Each left-hand side must contain at least one nonterminal symbol.

Derivation process may be desribed using $\xRightarrow[*]{}$ notation:
\begin{equation*}
\begin{split}
P &\xRightarrow[1]{} \bm{P} \; \bm{L} \\
  &\xRightarrow[2]{} P \; \bm{I} \; \bm{C} \; \bm{T} \\
  &\xRightarrow[3]{} P \; \textbf{\texttt{30}} \; C \; T \\
  &\xRightarrow[4]{} P \; \texttt{30} \; \textbf{\texttt{PRINT}} \; T \\
  &\xRightarrow[5]{} \bm{P} \; \bm{L} \; \texttt{30} \; \texttt{PRINT} \; T \\
  &\xRightarrow[6]{} \dots \\
\end{split}
\end{equation*}


\end{document}
