% (The MIT License)
%
% Copyright (c) 2022-2023 Yegor Bugayenko
%
% Permission is hereby granted, free of charge, to any person obtaining a copy
% of this software and associated documentation files (the 'Software'), to deal
% in the Software without restriction, including without limitation the rights
% to use, copy, modify, merge, publish, distribute, sublicense, and/or sell
% copies of the Software, and to permit persons to whom the Software is
% furnished to do so, subject to the following conditions:
%
% The above copyright notice and this permission notice shall be included in all
% copies or substantial portions of the Software.
%
% THE SOFTWARE IS PROVIDED 'AS IS', WITHOUT WARRANTY OF ANY KIND, EXPRESS OR
% IMPLIED, INCLUDING BUT NOT LIMITED TO THE WARRANTIES OF MERCHANTABILITY,
% FITNESS FOR A PARTICULAR PURPOSE AND NONINFRINGEMENT. IN NO EVENT SHALL THE
% AUTHORS OR COPYRIGHT HOLDERS BE LIABLE FOR ANY CLAIM, DAMAGES OR OTHER
% LIABILITY, WHETHER IN AN ACTION OF CONTRACT, TORT OR OTHERWISE, ARISING FROM,
% OUT OF OR IN CONNECTION WITH THE SOFTWARE OR THE USE OR OTHER DEALINGS IN THE
% SOFTWARE.

\documentclass{article}
\usepackage{../ppa}
\newcommand*\thetitle{Formal Semantics}
\newcommand*\thesubtitle{Rules, SOS, Natural, $\lambda$-calculus}
\begin{document}

\plush{\innoTitlePage{4}}

\pptToc

\pptSection{Inference Rule}

\plush{}

\pptSection{Rewrite Rule}

\plush{}

\plush{\pptChapter{Operational vs Denotational}}

The \emph{denotational semantics} assign to every expression the \emph{number} denoted by that expression:
\begin{gather*}
\bm{\Downarrow} \subseteq \mathcal{A} \times \mathcal{D} \\
x^n \bm{\Downarrow} y \quad\text{where}\; y = x \times x \times x \dots \times x \quad\text{($n$ times)}
\end{gather*}

The \emph{operational semantics} describe the computation steps taken in order to evaluate the expression to \emph{normal form}:
\begin{gather*}
\bm{\leadsto} \subseteq \mathcal{A} \times \mathcal{A} \\
x^n \bm{\leadsto} x \times x^{n-1} \quad\text{if}\; x > 0 \quad\quad x^0 \bm{\leadsto} 1
\end{gather*}

The operational semantics is the specification of an \emph{interpreter} for the programming language whereas the denotational semantics tries to capture the \emph{mathematically essence} of a program, abstracting away from computational details.

\plush{}

\pptSection{Natural Semantic (Denotational)}

Syntax: {\ttfamily FASTER; STOP; SLOWER;}\\
Semantic:
\begin{gather*}
\frac
  {}
  {\langle \texttt{STOP}, s \rangle \Downarrow \langle \texttt{tt}, 0 \rangle }
\\[24pt]
\frac
  { s < 60 }
  {\langle \texttt{FASTER}, s \rangle \Downarrow \langle \texttt{tt}, s + 20 \rangle }
\quad
\frac
  { s \geq 60 }
  {\langle \texttt{FASTER}, s \rangle \Downarrow \langle \texttt{ff}, s \rangle }
\\[24pt]
\frac
  {}
  {\langle \texttt{SLOWER}, s \rangle \Downarrow \langle \texttt{tt}, \text{max}(0, s - 20) \rangle }
\\[24pt]
\frac
  {\langle C_1, s \rangle \Downarrow \langle r_1, s' \rangle \quad \langle C_2, s' \rangle \Downarrow \langle r_2, s'' \rangle }
  {\langle C_1 ; C_2, s \rangle \Downarrow \langle r_1 \wedge r_2, s'' \rangle }
\end{gather*}
\plush{}

\pptSection{Proof Tree}

\pptSection{Structural Semantic (Operational)}

\plush{}

\pptSection{Literature}

Christopher Strachey (2000),
\emph{Fundamental Concepts in Programming Languages}

Alexander Kurz (2022),
\href{https://hackmd.io/@alexhkurz/Hkf6BTL6P}{Operational and Denotational Semantics}

Michael Pradel (2021),
\href{https://www.youtube.com/watch?v=jsBHd3-04oA}{Lectures on ``Operational Semantics''}

Gordon Plotkin (1981),
\href{https://web.eecs.umich.edu/~weimerw/2006-615/reading/plotkin81structural.pdf}{A Structural Approach to Operational Semantics}

\end{document}
