% (The MIT License)
%
% SPDX-FileCopyrightText: Copyright (c) 2022-2025 Yegor Bugayenko
% SPDX-License-Identifier: MIT

\documentclass{article}
\usepackage{../ppa}
\usepackage{setspace}
\newcommand*\thetitle{Model Checking}
\newcommand*\thesubtitle{Theory, SPIN, Java PathFinder}
\begin{document}

\lnTitlePage{9}{10}{jSuSo4JnYQI}

\pptToc

\plush{\pptChapter[Example]{Motivating Example}}

\pptSection{Div by Zero}

\begin{pptWide}{4}
{\small\begin{verbatim}
// Process no. 1:

extern int x;
extern double y;
int measure() {
  if (x != 0) {
    y = 1.0 / x;
  }
}
\end{verbatim}
}
\par\columnbreak\par
\scalebox{.8}{\begin{tikzpicture}[graph]
\node[draw=none,minimum width=0pt,inner sep=0pt] (start) {$\circ$};
\node[terminal,below=1cm of start] (neq) {\texttt{x != 0}}; \path (start) edge (neq);
\node[terminal,below right=1cm and .1cm of neq] (div) {\texttt{y = 1 / x}}; \path (neq) edge (div);
\node[draw=none,minimum width=0pt,inner sep=0pt,below=3cm of neq] (exit) {$\circ$};
\path (neq) edge (exit);
\path (exit) edge [color=orange, line width=2pt, bend left=90] (start);
\end{tikzpicture}}
\par\columnbreak\par
{\small\begin{verbatim}
// Process no. 2:

extern int x;
void roll() {
  x += 1;
  if (x > 10) {
    x -= 10;
  }
}
\end{verbatim}
}
\par\columnbreak\par
\scalebox{.8}{\begin{tikzpicture}[graph]
\node[draw=none,minimum width=0pt,inner sep=0pt] (start) {$\circ$};
\node[terminal,below=1cm of start] (plus) {\texttt{x = x + 1}}; \path (start) edge (plus);
\node[terminal,below=1cm of plus] (gt) {\texttt{x > 10}}; \path (plus) edge (gt);
\node[terminal,below right=2cm and 1.2cm of gt,anchor=center] (set) {\texttt{x = x - 10}}; \path (gt) edge (set);
\node[draw=none,minimum width=0pt,inner sep=0pt,below=3cm of gt] (exit) {$\circ$};
\path (gt) edge (exit);
\path (exit) edge [color=orange, line width=2pt, bend left=90] (start);
\end{tikzpicture}}
\end{pptWide}

Can we detect ``division by zero'' using symbolic execution? Is ``division by zero'' the only error here?

\plush{}

\pptSection[ProMeLa]{ProMeLa (Process Meta Language)}

\begin{pptWide}{2}
{\small\begin{verbatim}
extern int x;
extern double y;
int measure() {
  if (x != 0) {
    y = 1.0 / x;
  }
}
void roll() {
  x += 1;
  if (x > 10) {
    x -= 10;
  }
}
\end{verbatim}
}
\par\columnbreak\par
{\scriptsize\begin{verbatim}
int x; bool dbz;
active proctype measure() {
  do :: true ->
    if
    :: (x != 0) -> dbz = (x == 0)
    :: skip
    fi
  od
}
active proctype roll() {
  do :: true ->
    x = x + 1;
    if
    :: x > 10 -> x = x - 10
    :: skip
    fi
  od
}
\end{verbatim}
}
\end{pptWide}

\plush{}

\pptSection[SPIN]{SPIN (Simple ProMeLa Interpreter)}

\begin{pptWide}{2}
{\scriptsize\begin{verbatim}
int x; bool dbz;
active proctype measure() {
  do :: true ->
    if
    :: (x != 0) -> dbz = (x == 0)
    :: skip
    fi
  od
}
active proctype roll() {
  do :: true ->
    x = x + 1;
    if
    :: x > 10 -> x = x - 10
    :: skip
    fi;
    printf("x = %d\n", x);
  od
}
\end{verbatim}
}
\par\columnbreak\par
{\scriptsize\begin{verbatim}
$ spin main.pml | head
          x = 1
          x = 2
          x = 3
          x = 4
          x = 5
          x = 6
          x = 7
          x = 8
          x = 9
          x = 10
$ spin main.pml | tail
...
\end{verbatim}
}

Just checkout \href{https://github.com/nimble-code/Spin}{this repo} and run \texttt{make},
the \texttt{spin} binary will be compiled.
\end{pptWide}

\plush{}

\pptSection[Monitor]{Monitoring Process}

\begin{pptWide}{2}
{\scriptsize\begin{verbatim}
int x; bool dbz;
active proctype measure() {
  do :: true ->
    if
    :: (x != 0) -> dbz = (x == 0)
    :: skip
    fi
  od
}
active[2] proctype roll() {
  do :: true ->
    x = x + 1;
    if
    :: x > 10 -> x = x - 10
    :: skip
    fi
  od
}
\end{verbatim}
}
\par\columnbreak\par
{\scriptsize\begin{verbatim}
active proctype monitor() {
  do :: true ->
    assert(!dbz);
    assert(x >= 0);
  od
}
\end{verbatim}
}

Pay attention to the \texttt{[2]} suffix after the \texttt{active}
keyword. It tells SPIN to start two instances of the \texttt{roll} process.
\end{pptWide}

\plush{}

\pptSection[Assertion]{Fail on Assertion}

\begin{pptWide}{2}
{\scriptsize\begin{verbatim}
int x; bool dbz;
active proctype measure() {
  do :: true ->
    if
    :: (x != 0) -> dbz = (x == 0)
    :: skip
    fi
  od }
active[2] proctype roll() {
  do :: true ->
    x = x + 1;
    if
    :: x > 10 -> x = x - 10
    :: skip
    fi
  od }
active proctype monitor() {
  do :: true -> assert(!dbz); assert(x >= 0); od
}
\end{verbatim}
}
\par\columnbreak\par
{\scriptsize\begin{verbatim}
$ spin main.pml
spin: main.pml:22, Error: assertion violated
spin: text of failed assertion: assert((x>=0))
#processes: 4
    x = -9
    dbz = 0
584:  proc  3 (monitor:1) main.pml:22 (state 3)
584:  proc  2 (roll:1) main.pml:17 (state 7)
584:  proc  1 (roll:1) main.pml:18 (state 9)
584:  proc  0 (measure:1) main.pml:9 (state 8)
4 processes created
\end{verbatim}
}
\end{pptWide}

\plush{}

\plush{\pptChapter[Theory]{The Theory}}

\pptSection[Idea]{The Idea}

\emph{Model checking} is a method for checking whether a finite-state model of a system meets a given specification.

\begin{enumerate}
\item Represent software as a \emph{model}
\item Define \emph{constraints} on the model (using temporal logic)
\item Evaluate the model until constraints are \emph{violated/met}
\item Refine the model and constraints
\end{enumerate}

\plush{}

\pptSection[Model]{The Model}

\pptPic{0.8}{spin.png}

{\scriptsize The picture is taken from ``The Model Checker SPIN'' paper by Gerard J. Holzmann.}

\plush{}

\pptSection[LTL]{Linear Temporal Logic}

\pptPic{.9}{ltl.png}

\plush{}

\plush{\pptChapter[Model-less]{Model-less Model Checking}}

\pptSection{Race Condition}

A \emph{race condition} is the condition of where the system's substantive behavior is dependent on the sequence
or timing of other uncontrollable events.

\begin{pptWide}{2}
{\scriptsize\begin{verbatim}
public class Race {
  static int d = 42;
  public static void main (String[] args)
    throws Exception {
    new Thread(
      () -> {
        d = 0;
      }
    ).start();
    System.out.printf("x = %d\n", 420 / d);
  }
}
\end{verbatim}
}
\par\columnbreak\par
{\scriptsize\begin{verbatim}
$ javac Race.java
$ while true; do java Race; done
x = 10
x = 10
x = 10
x = 10
Exception in thread "main"
  java.lang.ArithmeticException: / by zero
  at Race.main(Race.java:9)
x = 10
x = 10
^C
\end{verbatim}
}
\end{pptWide}
\plush{}

\pptSection[Explosion]{States and Their Explosion}

\begin{pptWide}{2}
{\scriptsize\begin{verbatim}
public class Race {
  static int d = 42;
  public static void main (String[] args)
    throws Exception {
    new Thread(
      () -> {
        d = 0;
      }
    ).start();
    System.out.printf("x = %d\n", 420 / d);
  }
}
\end{verbatim}
}
\par\columnbreak\par
\scalebox{.8}{\begin{tikzpicture}[graph]
\node[draw=none,minimum width=0pt,inner sep=0pt] (start) {$\circ$};
\node[terminal,below=1.5cm of start] (set) {\texttt{d $\to$ 42}}; \path (start) edge (set);
\node[right=0cm of set,draw=none,color=orange] {$s_1$};
\node[terminal,below left=1.5cm of set] (zero) {\texttt{d $\to$ 0}}; \path (set) edge (zero);
\node[terminal,below=1.5cm of zero,draw=none] (error) {\texttt{error}}; \path (zero) edge (error);
\node[right=0cm of zero,draw=none,color=orange] {$s_2$};
\node[terminal,below right=1.5cm of set,draw=none] (printf) {\texttt{printf}}; \path (set) edge (printf);
\node[draw=none,minimum width=0pt,inner sep=0pt,below=1.5cm of printf] (exit) {$\circ$}; \path (printf) edge (exit);
\end{tikzpicture}}
\end{pptWide}

As the number of \emph{state variables} in the system increases, the size of the system state space
grows exponentially. This is called the \emph{state explosion problem}.

\plush{}

\pptSection[JPF]{Java PathFinder}

\begin{pptWide}{2}
{\setstretch{0.8}\fontsize{10}{10}\selectfont\begin{verbatim}
$ java -jar build/RunJPF.jar src/examples/Race.jpf
JavaPathfinder core system v8.0 (rev 3408119d115e539956a3d920e22e856e05bb9d23)
- (C) 2005-2014 United States Government. All rights reserved.


==================================== system under test
Race.main()

==================================== search started: 4/21/23 5:43 AM
x = 10

==================================== error 1
gov.nasa.jpf.listener.PreciseRaceDetector
race for field Race.d
  main at Race.main(Race.java:9)
    "System.out.printf("x = %d\n", 420 / d);"  READ:  getstatic Race.d
  Thread-1 at Race.lambda$main$0(Race.java:6)
    "d = 0;"  WRITE: putstatic Race.d


==================================== trace #1
------------------------------------------------------ transition #0 thread: 0
gov.nasa.jpf.vm.choice.ThreadChoiceFromSet {id:"ROOT" ,1/1,isCascaded:false}
      [6345 insn w/o sources]
  Race.java:2                    : static int d = 42;
  Race.java:1                    : public class Race {
      [1 insn w/o sources]
  Race.java:4                    : new Thread(
      [145 insn w/o sources]
  Race.java:8                    : ).start();
      [1 insn w/o sources]
------------------------------------------------------ transition #1 thread: 1
gov.nasa.jpf.vm.choice.ThreadChoiceFromSet {id:"START" ,2/2,isCascaded:false}
      [3 insn w/o sources]
  Race.java:6                    : d = 0;
------------------------------------------------------ transition #2 thread: 0
gov.nasa.jpf.vm.choice.ThreadChoiceFromSet {id:"SHARED_CLASS" ,1/2,isCascaded:false}
      [2 insn w/o sources]
  Race.java:9                    : System.out.printf("x = %d\n", 420 / d);
------------------------------------------------------ transition #3 thread: 0
gov.nasa.jpf.vm.choice.ThreadChoiceFromSet {id:"SHARED_CLASS" ,1/2,isCascaded:false}
  Race.java:9                    : System.out.printf("x = %d\n", 420 / d);

=================================== results
error #1: gov.nasa.jpf.listener.PreciseRaceDetector
  "race for field Race.d   main at Race.main(Race.jav..."

=================================== statistics
elapsed time:       00:00:00
states:             new=6,visited=0,backtracked=2,end=1
search:             maxDepth=4,constraints=0
choice generators:  thread=5 (signal=0,lock=1,sharedRef=2,...
heap:               new=741,released=22,maxLive=722,gcCycles=4
instructions:       6546
max memory:         491MB
loaded code:        classes=83,methods=1817

================================== search finished: 4/21/23 5:43 AM
\end{verbatim}
}
\end{pptWide}

\plush{}

\pptSection[ChatGPT]{What about ChatGPT?}

\begin{pptWide}{2}
\pptPic{.7}{gpt1.png}
\par\columnbreak\par
\pptPic{.7}{gpt2.png}
\end{pptWide}

\plush{}
\plush{\pptChapter[Literature]{Further Reading/Watching}}

Introduction lecture by \href{https://www.youtube.com/watch?v=VHWEldcSx14}{Joost-Pieter Katoen}

\href{https://spinroot.com/spin/Doc/p40-ben-ari.pdf}{A Primer on Model Checking} by Mordechai Ben-Ari

\href{http://spinroot.com/spin/Doc/ieee97.pdf}{The Model Checker SPIN} by Gerard J. Holzmann

\end{document}
