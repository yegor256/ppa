% SPDX-FileCopyrightText: Copyright (c) 2022-2025 Yegor Bugayenko
% SPDX-License-Identifier: MIT

\documentclass{article}
\usepackage{../ppa}
\newcommand*\thetitle{Contextual Analysis}
\newcommand*\thesubtitle{AST, Interpretation, CFG}
\begin{document}

\lnTitlePage{3}{10}{fBcQyX_wAhQ}

\pptToc

\pptBanner{Code Understanding Pipeline}
\begin{tikzpicture}[graph]
\node[draw=none] (start) {};
\node[terminal,below right=.4cm and -2cm of start] (parsing) {Lexical Analysis}; \draw (start) edge node[note,anchor=east] {Chars} (parsing);
\node[terminal,below right=.6cm and -2cm of parsing] (syntax) {Syntactical Analysis}; \draw (parsing) edge node[note,anchor=east] {Tokens} (syntax);
\node[terminal,below right=.6cm and -2cm of syntax] (inter) {Interpretation}; \draw (syntax) edge node[note,anchor=west] {Concrete Syntax Tree (CST)} (inter);
\node[terminal,below right=.6cm and -2cm of inter] (context) {Contextual Analysis}; \draw (inter) edge node[note,anchor=west] {Abstract Syntax Tree (AST)} (context);
\node[terminal,below right=.6cm and -2cm of context] (static) {Static Analysis}; \draw (context) edge node[note,anchor=east] {Decorated AST} (static);
\node[draw=none,below right=.4cm and -2cm of static] (end) {}; \draw (static) edge node[note,anchor=east] {Errors} (end);
\end{tikzpicture}
\plush{}

\plush{\pptChapter[Trees]{Concrete vs. Abstract}}

The \emph{concrete syntax} of a programming language is defined by a context free grammar (CFG).
The \emph{abstract syntax} of an implementation is the set of trees used to represent programs in the implementation.

\plush{}

Simple program:
\begin{ffcode}
PRINT "Hi," + name;
EXIT;
\end{ffcode}

\begin{multicols}{2}
Concrete Syntax Tree:

\scalebox{.6}{\begin{tikzpicture}[graph]
\node (p) {$P$};
\node[below left=1cm and 4cm of p] (l1) {$L$}; \path (p) edge (l1);
\node[below left=1cm and 2cm of l1] (print) {$C$}; \path (l1) edge (print);
\node[terminal,below=1cm of print] (print-t) {\texttt{PRINT}}; \path (print) edge (print-t);
\node[below left=1cm and 0cm of l1] (hello) {$S$}; \path (l1) edge (hello);
\node[terminal,below=3cm of hello] (hello-t) {\texttt{"Hi,"}}; \path (hello) edge (hello-t);
\node[below right=1cm and 0cm of l1] (plus) {$O$}; \path (l1) edge (plus);
\node[terminal,below=1cm of plus] (plus-t) {\texttt{+}}; \path (plus) edge (plus-t);
\node[below right=1cm and 2cm of l1] (name) {$V$}; \path (l1) edge (name);
\node[terminal,below=3cm of name] (name-t) {\texttt{name}}; \path (name) edge (name-t);
\node[terminal,below right=1cm and 5cm of l1] (semi1) {\texttt{";"}}; \path (l1) edge (semi1);
%
\node[below right=1cm and 6cm of p] (l2) {$L$}; \path (p) edge (l2);
\node[below left=1cm of l2] (exit) {$C$}; \path (l2) edge (exit);
\node[terminal,below=1cm of exit] (exit-t) {\texttt{EXIT}}; \path (exit) edge (exit-t);
\node[terminal,below right=1cm of l2] (semi2) {\texttt{";"}}; \path (l2) edge (semi2);
\end{tikzpicture}}
\par\columnbreak\par
Abstract Syntax Tree:

\scalebox{.75}{\begin{tikzpicture}[graph]
\node[terminal] (p) {program};
\node[terminal,below left=1cm of p] (print) {print}; \path (p) edge (print);
\node[terminal,below right=1cm of p] (exit) {exit}; \path (p) edge (exit);
\node[terminal,below=1cm of print] (plus) {plus}; \path (print) edge (plus);
\node[terminal,below left=1cm of plus] (a1) {"Hi,"}; \path (plus) edge (a1);
\node[terminal,below right=1cm of plus] (a2) {name}; \path (plus) edge (a2);
\end{tikzpicture}}
\end{multicols}
\plush{}

\plush{\pptChapter{Identification}}

\begin{multicols}{2}
{\ttfamily
int x; \\
loop \{ int x; x++; \}; \\
print x;}
\par\columnbreak\par
\begin{tikzpicture}[graph]
\node[terminal] (p) {program};
\node[terminal,below left=1cm of p] (var1) {int x}; \path (p) edge (var1);
\node[terminal,below=1.2cm of p] (loop) {loop}; \path (p) edge (loop);
\node[terminal,below right=1cm and 3cm of p] (print) {print}; \path (p) edge (print);
\node[terminal,below=1cm of print] (x2) {x}; \path (print) edge (x2);
\node[terminal,below left=1cm of loop] (var2) {int x}; \path (loop) edge (var2);
\node[terminal,below right=1cm of loop] (plus) {++}; \path (loop) edge (plus);
\node[terminal,below=1cm of plus] (x1) {x}; \path (plus) edge (x1);
\end{tikzpicture}
\end{multicols}

Somehow we must \emph{link} different $x$ to different places, where they are \emph{declared},
maybe with the help of "\emph{Identification Table}," or by attaching attributes to AST nodes, or both.
We may want to track their \emph{indentation levels}.
\plush{}

\plush{\pptChapter{Static Type Checking}}

Dynamically-typed languages perform \emph{type checking} at \emph{runtime},
while statically typed languages perform type checking at \emph{compile time}.

\begin{multicols}{2}
{\ttfamily
var x = "Sofi"; \\
loop \{ var x; x++; \}; \\
print "Hello," + x;}
\par\columnbreak\par
\begin{tikzpicture}[graph]
\node[terminal] (p) {program};
\node[terminal,below left=1cm of p] (var1) {x="Sofi"}; \path (p) edge (var1);
\node[terminal,below=1.2cm of p] (loop) {loop}; \path (p) edge (loop);
\node[terminal,below left=1cm of loop] (var2) {var x}; \path (loop) edge (var2);
\node[terminal,below right=1cm of loop] (plusplus) {++}; \path (loop) edge (plus);
\node[terminal,below=1cm of plusplus] (x1) {x}; \path (plusplus) edge (x1);
\node[terminal,below right=3cm and 3cm of p] (print) {print}; \path (p) edge (print);
\node[terminal,below=1cm of print] (plus) {+}; \path (print) edge (plus);
\node[terminal,below left=1cm of plus] (hello) {"Hello,"}; \path (plus) edge (hello);
\node[terminal,below right=1cm of plus] (x2) {x}; \path (plus) edge (x2);
\end{tikzpicture}
\end{multicols}

\plush{}

\plush{\pptChapter{AST Visitor}}

ANTLR4 lets us implement the following interface:

\begin{ffcode}
public interface ParseTreeListener {
  void visitTerminal(TerminalNode var1);
  void visitErrorNode(ErrorNode var1);
  void enterEveryRule(ParserRuleContext var1);
  void exitEveryRule(ParserRuleContext var1);
}
\end{ffcode}

\plush{}

Then:
\begin{ffcode}
MyLexer lexer = new MyLexer(text); // Lexer
MyParser parser = new MyParser(
  new CommonTokenStream(lexer)  // Parser
);
MyListener lsr = new MyListener(); // ParseTreeListener
new ParseTreeWalker().walk(lsr, parser.program());
\end{ffcode}

\plush{}

\plush{\pptChapter{Decorated AST}}

\begin{multicols}{2}
{\ttfamily
int x; \\
loop \{ int x; x++; \}; \\
print x;}
\par\columnbreak\par
\begin{tikzpicture}[graph]
\node[terminal] (p) {program};
\node[terminal,below left=1cm of p] (var1) {int x}; \path (p) edge (var1);
\node[terminal,below=1.2cm of p] (loop) {loop}; \path (p) edge (loop);
\node[terminal,below right=1cm and 3cm of p] (print) {print}; \path (p) edge (print);
\node[terminal,below=1cm of print] (x2) {x}; \path (print) edge (x2);
\node[terminal,below left=1cm of loop] (var2) {int x}; \path (loop) edge (var2);
\node[terminal,below right=1cm of loop] (plus) {++}; \path (loop) edge (plus);
\node[terminal,below=1cm of plus] (x1) {x}; \path (plus) edge (x1);
\node[terminal,draw=none,below=0cm of var1,font={\color{red}}] {\texttt{i31}};
\node[terminal,draw=none,below=0cm of var2,font={\color{red}}] {\texttt{i32}};
\node[terminal,draw=none,below=0cm of x1,font={\color{red}}] {\texttt{i32,int}};
\node[terminal,draw=none,below=0cm of x2,font={\color{red}}] {\texttt{i31,int}};
\end{tikzpicture}
\end{multicols}
\plush{}

\plush{\pptChapter[CFG]{Control Flow Graph}}

\begin{multicols}{2}
{\ttfamily
int x = 42; \\
loop \{ int x = 0; x++; \}; \\
print x;}
\par\columnbreak\par
\begin{tikzpicture}[graph]
\node (start) {$\circ$};
\node[terminal,below=1cm of start] (var1) {int x = 42;}; \path (start) edge (var1);
\node[terminal,below=1cm of var1] (var2) {int x = 0;}; \path (var1) edge (var2);
\node[terminal,below=1cm of var2] (plus) {x++;}; \path (var2) edge (plus);
\node[terminal,below=1cm of plus] (print) {print x;}; \path (plus) edge (print);
\node[below=1cm of print] (exit) {$\circ$}; \path (print) edge (exit);
\path (plus.east) edge [bend right=90] (var2.east);
\end{tikzpicture}
\end{multicols}


\end{document}
