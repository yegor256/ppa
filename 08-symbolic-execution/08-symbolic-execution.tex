% (The MIT License)
%
% Copyright (c) 2022-2023 Yegor Bugayenko
%
% Permission is hereby granted, free of charge, to any person obtaining a copy
% of this software and associated documentation files (the 'Software'), to deal
% in the Software without restriction, including without limitation the rights
% to use, copy, modify, merge, publish, distribute, sublicense, and/or sell
% copies of the Software, and to permit persons to whom the Software is
% furnished to do so, subject to the following conditions:
%
% The above copyright notice and this permission notice shall be included in all
% copies or substantial portions of the Software.
%
% THE SOFTWARE IS PROVIDED 'AS IS', WITHOUT WARRANTY OF ANY KIND, EXPRESS OR
% IMPLIED, INCLUDING BUT NOT LIMITED TO THE WARRANTIES OF MERCHANTABILITY,
% FITNESS FOR A PARTICULAR PURPOSE AND NONINFRINGEMENT. IN NO EVENT SHALL THE
% AUTHORS OR COPYRIGHT HOLDERS BE LIABLE FOR ANY CLAIM, DAMAGES OR OTHER
% LIABILITY, WHETHER IN AN ACTION OF CONTRACT, TORT OR OTHERWISE, ARISING FROM,
% OUT OF OR IN CONNECTION WITH THE SOFTWARE OR THE USE OR OTHER DEALINGS IN THE
% SOFTWARE.

\documentclass{article}
\usepackage{../ppa}
\newcommand*\thetitle{Symbolic Execution}
\newcommand*\thesubtitle{...}
\begin{document}

\plush{\innoTitlePage{8}}

\pptToc

\plush{\pptChapter{In Theory}}

\pptSection[CFG]{Control Flow Graph}

\begin{multicols}{2}
{\ttfamily
int f(int x) \{ \\
\quad  int a = x * 2; \\
\quad  a = a - 4; \\
\quad  if (a == 0) \\
\quad\quad error("Div by zero!"); \\
\quad  return 42 / a; \\
\}}
\par\columnbreak\par
\scalebox{.9}{\begin{tikzpicture}[graph]
\node[draw=none,minimum width=0pt,inner sep=0pt] (start) {$\circ$};
\node[terminal,below=1cm of start] (mul) {\texttt{a := x * 2}}; \path (start) edge (mul);
\node[terminal,below=1cm of mul] (minus) {\texttt{a := a - 4}}; \path (mul) edge (minus);
\node[terminal,below=1cm of minus] (if) {\texttt{a == 0}}; \path (minus) edge (if);
\node[terminal,below right=2cm of if] (error) {\texttt{error}}; \path (if) edge (error);
\node[terminal,below=3cm of if] (return) {\texttt{42 / a}};
\path (if) edge [bend right=30] (return);
\end{tikzpicture}}
\end{multicols}

\plush{}

\pptSection[Feasibility]{Path Feasibility}

A path is \emph{feasible} if there exists an input $\mathcal{I}$ to the program that covers the path; i.e., when program is executed with $\mathcal{I}$ as input, the path is taken.

\begin{multicols}{2}
{\begin{verbatim}
int f(int x) {
  int a = x * 2;
  a = a - 4;
  if (a == 0)
    error("Div by zero!");
  return 42 / a;
}
\end{verbatim}
}
\par\columnbreak\par
\scalebox{.8}{\begin{tikzpicture}[graph]
\node[draw=none,minimum width=0pt,inner sep=0pt] (start) {$\circ$};
\node[terminal,below=1cm of start] (mul) {\texttt{a := x * 2}}; \path (start) edge[line width=5pt] (mul);
\node[terminal,below=1cm of mul] (minus) {\texttt{a := a - 4}}; \path (mul) edge[line width=5pt] (minus);
\node[terminal,below=1cm of minus] (if) {\texttt{a == 0}}; \path (minus) edge[line width=5pt] (if);
\node[terminal,below right=2cm of if] (error) {\texttt{error}}; \path (if) edge[line width=5pt] (error);
\node[terminal,below=3cm of if] (return) {\texttt{42 / a}};
\path (if) edge [bend right=30] (return);
\end{tikzpicture}}
\end{multicols}

\plush{}

\pptSection[Infeasible]{Infeasible Path}

A path is \emph{infeasible} if there exists no input $\mathcal{I}$ that covers the path.

\begin{multicols}{2}
{\begin{verbatim}
int f(int x) {
  int a = x * x;
  if (a < 0)
    error("Too small!");
  return 42 / a;
}
\end{verbatim}
}
\par\columnbreak\par
\scalebox{1}{\begin{tikzpicture}[graph]
\node[draw=none,minimum width=0pt,inner sep=0pt] (start) {$\circ$};
\node[terminal,below=1cm of start] (mul) {\texttt{a := x * x}}; \path (start) edge[line width=5pt] node[note,right] {a} (mul);
\node[terminal,below=1cm of mul] (if) {\texttt{a < 0}}; \path (mul) edge[line width=5pt] node[note,right] {b} (if);
\node[terminal,below right=2cm of if] (error) {\texttt{error}}; \path (if) edge[line width=5pt] node[note,right] {c} (error);
\node[terminal,below=3cm of if] (return) {\texttt{42 / a}};
\path (if) edge [bend right=30] node[note,right] {d} (return);
\end{tikzpicture}}
\end{multicols}

\plush{}

\pptSection{Symbols}

\begin{multicols}{2}
{\begin{verbatim}
int f(int x) {
  int a = x * 2;
  a = a - 4;
  if (a == 0)
    error("Div by zero!");
  return 42 / a;
}
\end{verbatim}
}
\par\columnbreak\par
\scalebox{.9}{\begin{tikzpicture}[graph]
\node[draw=none,minimum width=0pt,inner sep=0pt] (start) {$\circ$};
\node[terminal,below=1cm of start] (mul) {\texttt{a := x * 2}}; \path (start) edge[line width=5pt] node[note,right] {$x \mapsto X$} (mul);
\node[terminal,below=1cm of mul] (minus) {\texttt{a := a - 4}}; \path (mul) edge[line width=5pt] node[note,right] {$x \mapsto X, a \mapsto 2X$} (minus);
\node[terminal,below=1cm of minus] (if) {\texttt{a == 0}}; \path (minus) edge[line width=5pt] node[note,right] {$x \mapsto X, a \mapsto 2X - 4$} (if);
\node[terminal,below right=2cm of if] (error) {\texttt{error}}; \path (if) edge[line width=5pt] node[note,right=1cm] {$x \mapsto X, a \mapsto 2X - 4$} (error);
\node[terminal,below=3cm of if] (return) {\texttt{42 / a}};
\path (if) edge [bend right=30] (return);
\end{tikzpicture}}
\end{multicols}

\plush{}

\pptSection[PC]{Path Conditions}

\emph{Path condition} is a condition on the input symbols such that if a path is feasible its path-condition is satisfiable.

\begin{multicols}{2}
{\begin{verbatim}
int f(int x) {
  int a = x * 2;
  a = a - 4;
  if (a == 0)
    error("Div by zero!");
  return 42 / a;
}
\end{verbatim}
}
\par\columnbreak\par
\scalebox{.8}{\begin{tikzpicture}[graph]
\node[draw=none,minimum width=0pt,inner sep=0pt] (start) {$\circ$};
\node[terminal,below=1cm of start] (mul) {\texttt{a := x * 2}}; \path (start) edge[line width=5pt] node[note,right] {$x \mapsto X$} node[note,left] {$\textbf{tt}$} (mul);
\node[terminal,below=1cm of mul] (minus) {\texttt{a := a - 4}}; \path (mul) edge[line width=5pt] node[note,right] {$x \mapsto X, a \mapsto 2X$} node[note,left] {$\textbf{tt}$} (minus);
\node[terminal,below=1cm of minus] (if) {\texttt{a == 0}}; \path (minus) edge[line width=5pt] node[note,right] {$x \mapsto X, a \mapsto 2X - 4$} node[note,left] {$\textbf{tt}$} (if);
\node[terminal,below right=2cm of if] (error) {\texttt{error}}; \path (if) edge[line width=5pt] node[note,right=1cm] {$x \mapsto X, a \mapsto 2X - 4$} node[note,left=1cm] {$2X - 4 = 0$} (error);
\node[terminal,below=3cm of if] (return) {\texttt{42 / a}};
\path (if) edge [bend right=30] (return);
\end{tikzpicture}}
\end{multicols}

\plush{}

\pptSection[Solver]{Constraint Solver}

A \emph{constraint solver} is a tool that finds satisfying assignments for a \emph{constraint}, if it is satisfiable.

A \emph{solution} of the constraint is a set of assignments, one for each free variable that makes the constraint satisfiable.

Constraint:
\begin{gather*}
x \mapsto X, \; a \mapsto 2X - 4 \\
2X - 4 = 0
\end{gather*}

Solution:
\begin{gather*}
X = 2
\end{gather*}

\plush{}

\plush{\pptChapter{In Practice}}

\pptSection[SAT]{SAT Solvers}

\emph{SAT solver} is a computer program which aims to solve the \emph{Boolean satisfiability problem}: whether the variables of a given Boolean formula can be consistently replaced by the values TRUE or FALSE in such a way that the formula evaluates to TRUE.

Examples:

\begin{equation*}
\begin{split}
a \wedge b \to \dots \\
a \wedge b \wedge \neg a \to \dots \\
a \vee b \vee \neg a \to \dots \\
a \wedge ( \textbf{ff} \vee \textbf{tt} ) \to \dots \\
\end{split}
\end{equation*}

All expressions are in Boolean logic.

\plush{}

\pptSection[SMT]{SMT Solvers}

\emph{SMT solver} is a computer program which aims to solve the \emph{satisfiability modulo theories}: determine whether a mathematical formula is satisfiable.

Examples:

\begin{equation*}
\begin{split}
a < 5 \wedge a > 3 \to \dots \\
a < 5 \wedge f(a) > 42 \to \dots \\
a < 5 \vee a > 10 \vee \neg a \to \dots \\
a \wedge \textbf{ff} \wedge x = 7 \to \dots \\
\end{split}
\end{equation*}

SMT solvers: Z3, cvc5, Yices, and many more...

\plush{}

\pptSection[Explosion]{Path Explosion}

\emph{Path explosion} refers to the fact that the number of control-flow paths in a program grows exponentially with an increase in program size and can even be infinite in the case of programs with unbounded loop iterations.

\begin{pptWide}{2}
{\small\begin{verbatim}
int a = 0;
for (int i = 10; i >= 0; i--) {
  a += 42 / i;
}
\end{verbatim}
}
\par\columnbreak\par
\scalebox{.7}{\begin{tikzpicture}[graph]
\node[draw=none,minimum width=0pt,inner sep=0pt] (start) {$\circ$};
\node[terminal,below=1cm of start] (zero) {\texttt{a := 0}}; \path (start) edge node[note,right] {a} (zero);
\node[terminal,below=1cm of zero] (for) {\texttt{i := 10}}; \path (zero) edge node[note,right] {b} (for);
\node[terminal,below=1cm of for] (gte) {\texttt{i >= 0}}; \path (for) edge node[note,right] {c} (gte);
\node[terminal,below right=1cm and 3cm of gte] (div) {\texttt{a += 42 / i}}; \path (gte) edge node[note,right] {d} (div);
\node[terminal,below left=1cm and 2cm of div] (minus) {\texttt{i-{}-}}; \path (div) edge node[note,right] {e} (minus);
\node[terminal,right=2cm of div] (error) {\texttt{error!}}; \path (div) edge node[note,right] {f} (error);
\node[draw=none,minimum width=0pt,inner sep=0pt, below left=3cm and 3cm of gte] (exit) {$\circ$};
\path (minus) edge[bend left=30] node[note,right] {g} (gte);
\path (gte) edge[bend right=30] node[note,right] {h} (exit);
\end{tikzpicture}}
\end{pptWide}

\plush{}

\pptSection[ChatGPT]{Clang Tidy vs. ChatGPT}

\begin{multicols}{2}
\pptPic{.85}{loop-tidy.png}
\par\columnbreak\par
\pptPic{0.9}{loop-gpt.png}
\end{multicols}

\plush{}

\pptSection[CSA]{Clang Tidy vs. Clang Static Analyzer}

{\scriptsize\begin{verbatim}
$ cat a.cpp
int main() {
  int a = 0;
  for (int i = 10; i >= 0; i--) {
    a += 42 / i;
  }
  return a;
}
$ clang-tidy a.cpp --
$ clang --analyze -Xclang -analyzer-constraints=z3 -Xclang -analyzer-max-loop -Xclang 5 a.cpp
$ clang --analyze -Xclang -analyzer-constraints=z3 -Xclang -analyzer-max-loop -Xclang 15 a.cpp
a.cpp:4:13: warning: Division by zero [core.DivideZero]
    a += 42 / i;
         ~~~^~~
1 warning generated.
\end{verbatim}

\plush{}

\plush{\pptChapter{Test Generation}}

\pptSection[Input]{Symbolic Input}

\begin{pptWide}{2}
{\scriptsize\begin{verbatim}
#include <climits>
#include "stdlib.h"
int f(int x) {
  int a = x * 2;
  a = a - 4;
  if (a == 0)
    exit(-1);
  return 42 / a;
}
int main(int argc, char** argv) {
  int x = atoi(argv[1]);
  return f(x);
}
\end{verbatim}
}
\par\columnbreak\par
{\ttfamily\scriptsize
\#include <climits> \\
\#include "stdlib.h" \\
{\color{red}\#include "klee/klee.h"} \\
int f(int x) \{ \\
\quad  int a = x * 2; \\
\quad  a = a - 4; \\
\quad  if (a == 0) \\
\quad\quad    exit(-1); \\
\quad  return 42 / a; \\
\} \\
int main(int argc, char** argv) \{ \\
{\color{red}
\quad  int x; \\
\quad  klee\_make\_symbolic(\&x, sizeof(x), "x");} \\
\quad  return f(x); \\
\}}
\end{pptWide}

\plush{}

\pptSection[Bitcode]{Compile to LLVM Bitcode}

{\scriptsize\begin{verbatim}
$ clang -I /opt/homebrew/Cellar/klee/2.3\_4/include -c -g \
  -emit-llvm -O0 -Xclang -disable-O0-optnone a.cpp

$ klee a.bc
KLEE: output directory is "/code/tmp/cpp/klee-out-2"
KLEE: Using STP solver backend
KLEE: done: total instructions = 38
KLEE: done: completed paths = 2
KLEE: done: partially completed paths = 0
KLEE: done: generated tests = 2

$ ls -al klee-out-0/*.ktest
-rw-r--r--   1 yb  staff    46 Apr  7 17:30 test000001.ktest
-rw-r--r--   1 yb  staff    46 Apr  7 17:30 test000002.ktest
\end{verbatim}
}

\plush{}

\pptSection[TCs]{Test Cases}

\begin{pptWide}{3}
{\tiny\begin{verbatim}
#include <climits>
#include "stdlib.h"
#include "klee/klee.h"
int f(int x) {
  int a = x * 2;
  a = a - 4;
  if (a == 0)
    exit(-1);
  return 42 / a;
}
int main(int argc, char** argv) {
  int x;
  klee_make_symbolic(&x, sizeof(x), "x");
  return f(x);
}
\end{verbatim}
}
\par\columnbreak\par
{\tiny\begin{verbatim}
$ ktest-tool klee-last/test000001.ktest
ktest file : 'klee-last/test000001.ktest'
args       : ['a.bc']
num objects: 1
object 0: name: 'x'
object 0: size: 4
object 0: data: b'\x02\x00\x00\x00'
object 0: hex : 0x02000000
object 0: int : 2
object 0: uint: 2
object 0: text: ....
\end{verbatim}
}
\par\columnbreak\par
{\tiny\begin{verbatim}
$ ktest-tool klee-last/test000002.ktest
ktest file : 'klee-last/test000002.ktest'
args       : ['a.bc']
num objects: 1
object 0: name: 'x'
object 0: size: 4
object 0: data: b'\x00\x00\x00\x00'
object 0: hex : 0x00000000
object 0: int : 0
object 0: uint: 0
object 0: text: ....
\end{verbatim}
}
\end{pptWide}

\plush{}

\pptSection[Replay]{Replaying Test Cases}

{\scriptsize\begin{verbatim}
$ export LD_LIBRARY_PATH=/opt/homebrew/Cellar/klee/2.3_4/lib:$LD_LIBRARY_PATH

$ clang -I /opt/homebrew/Cellar/klee/2.3_4/include -L/opt/homebrew/Cellar/klee/2.3_4/lib \
  -lkleeRuntest -Xclang -disable-O0-optnone a.cpp

$ KTEST_FILE=klee-last/test000001.ktest ./a.out ; echo $?
255

$ KTEST_FILE=klee-last/test000002.ktest ./a.out ; echo $?
246
\end{verbatim}
}

\plush{}

\plush{\pptChapter{Concolic Execution}}

\pptSection[...]{...}

\plush{}

...

\end{document}
