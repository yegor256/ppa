% (The MIT License)
%
% Copyright (c) 2022-2023 Yegor Bugayenko
%
% Permission is hereby granted, free of charge, to any person obtaining a copy
% of this software and associated documentation files (the 'Software'), to deal
% in the Software without restriction, including without limitation the rights
% to use, copy, modify, merge, publish, distribute, sublicense, and/or sell
% copies of the Software, and to permit persons to whom the Software is
% furnished to do so, subject to the following conditions:
%
% The above copyright notice and this permission notice shall be included in all
% copies or substantial portions of the Software.
%
% THE SOFTWARE IS PROVIDED 'AS IS', WITHOUT WARRANTY OF ANY KIND, EXPRESS OR
% IMPLIED, INCLUDING BUT NOT LIMITED TO THE WARRANTIES OF MERCHANTABILITY,
% FITNESS FOR A PARTICULAR PURPOSE AND NONINFRINGEMENT. IN NO EVENT SHALL THE
% AUTHORS OR COPYRIGHT HOLDERS BE LIABLE FOR ANY CLAIM, DAMAGES OR OTHER
% LIABILITY, WHETHER IN AN ACTION OF CONTRACT, TORT OR OTHERWISE, ARISING FROM,
% OUT OF OR IN CONNECTION WITH THE SOFTWARE OR THE USE OR OTHER DEALINGS IN THE
% SOFTWARE.

\documentclass{article}
\usepackage{../ppa}
\usepackage{setspace}
\newcommand*\thetitle{Program Analysis with ML}
\newcommand*\thesubtitle{ML, Coding Companions, The Future}
\begin{document}

\plush{\ppaTitlePage{10}}

\pptToc

\plush{\pptChapter[ML]{How Machine Learning Works?}}

\pptSection[Determinism]{Deterministic Algorithm}

A \emph{deterministic} algorithm means that given a particular input, the algorithm will always produce the same output:

{\scriptsize\begin{verbatim}
String sayHello(int hours, int minutes) {
  if (hours > 4 && hours < 12) {
    return "Good morning!";
  } else if (hours > 12 && hours < 18) {
    return "Good afternoon!";
  } else if (hours > 18 && hours < 22) {
    return "Good evening!";
  }
  return "Good night!";
}
\end{verbatim}
}

The behavior of the \texttt{sayHello()} function is defined upfront by its creator.

\plush{}

\pptSection[Training]{Training a Model}

``The process of training an \emph{ML model} involves providing a \emph{learning algorithm} with \emph{training data}. The algorithm finds patterns in the training data that map the \emph{input data attributes} to the \emph{target attributes}, and it outputs an ML model that captures these patterns.'' (c) Amazon

\begin{multicols}{2}
{\scriptsize\begin{verbatim}
 8:35  Good morning!
 8:40  Good morning!
10:00  How are you?
11:55  Good afternoon!
13:18  Good day!
14:50  Good afternoon!
15:22  Good afternoon, Sir!
17:14  Good evening!
...
22:34  Evening!
23:50  Good night!
\end{verbatim}
}
\par\columnbreak\par
\begin{equation*}
f : H \times M \to G
\end{equation*}
{\scriptsize\begin{equation*}
\begin{split}
H &= \{ 0, 1, 2, \dots, 23 \} \\
M &= \{ 0, 1, 2, \dots, 59 \} \\
G &= \{ \texttt{"Good morning!"}, \\
  & \quad\texttt{"Good afternoon!"}, \\
  & \quad\texttt{"Good evening!"}, \\
  & \quad\texttt{"Good night!"} \} \\
\end{split}
\end{equation*}}
\end{multicols}

\plush{}

\pptSection{Embedding}

``An \emph{embedding} is a relatively low-dimensional space into which you can translate high-dimensional vectors.'' (c) Google

\begin{multicols}{3}
{\scriptsize\begin{verbatim}
Features:

 8:35  Good morning!
 8:40  Good morning!
10:00  How are you?
11:55  Good afternoon!
13:18  Good day!
14:50  Good afternoon!
15:22  Good afternoon, Sir!
17:14  Good evening!
...
22:34  Evening!
23:50  Good night!
\end{verbatim}
}
\par\columnbreak\par
{\scriptsize\begin{verbatim}
Embeddings:

(8, 30, "morning")
(8, 45, "morning")
NIL
(12, 00, "morning")
NIL
(14, 45, "afternoon")
(15, 30, "afternoon")
(17, 00, "evening")
...
(22, 30, "evening")
(23, 45, "night")
\end{verbatim}
}
\par\columnbreak\par
{\scriptsize\begin{verbatim}
Vectors:

(8, 30, 0)
(8, 45, 0)

(12, 00, 0)

(14, 45, 1)
(15, 30, 1)
(17, 00, 2)
...
(22, 30, 2)
(23, 45, 3)
\end{verbatim}
}
\end{multicols}

\plush{}

\pptSection[Tasks]{Data Science}

There are five tasks to complete \emph{repeatedly}:

\begin{enumerate}
  \item Collect and clean a dataset
  \item Define features + embeddings
  \item Choose and tune an algorithm
  \item Train a Model
  \item Validate the Model
\end{enumerate}

\plush{}

\plush{\pptChapter[Analysis]{How Program Analysis Fits In?}}

\pptSection[Vectorization]{Code Vectorization}

...

\plush{}

\pptSection[Attributes]{Target Attributes}

...

\plush{}

\plush{\pptChapter[Companions]{AI Coding Companions}}

\pptSection[Complete]{Auto Code Completion}

This is how \href{https://github.com/features/copilot}{Copilot} by GitHub is suggesting code completion in our \href{https://www.eolang.org}{own programming language}, which he definitely hasn't seen before:

\begin{multicols}{4}
\pptPic{.9}{copilot-1.png}
\pptPic{.9}{copilot-2.png}
\pptPic{.9}{copilot-3.png}
\pptPic{.9}{copilot-4.png}
\end{multicols}

Read also about \href{https://aws.amazon.com/codewhisperer/}{AWS CodeWhisperer} at \href{https://www.allthingsdistributed.com/2023/04/how-ai-coding-companions-will-change-the-way-developers-work.html}{Werner Vogels' blog}. Also, about \href{https://github.com/codota}{TabNine} (used to be Codota).

\plush{}

\pptSection[PR]{Pull Request Explanation}

\pptPic{0.75}{pr-codex.png}

\href{https://github.com/decentralizedlabs/pr-codex}{PR-Codex} plugin for GitHub by \href{https://www.dlabs.app/}{dlabs}

\plush{}

\pptSection[Review]{Reviewing Changes}

\pptPic{0.75}{codex-qa.png}

The discussion happened in this GitHub issue: \href{https://github.com/objectionary/eo/pull/2034}{objectionary/eo:2034}

\plush{}

\pptSection[Risks]{Pull Request Risk Analysis}


\plush{}

\pptSection[Explain]{Explain This to Me!}

\pptPic{0.9}{mintlify.png}

\href{https://github.com/mintlify/writer}{Writer} plugin for VS-Code by \href{Mintlify}{https://writer.mintlify.com/}

\plush{}

\pptSection[Repeat]{Repeat After Me!}

Read about ``\href{https://devblogs.microsoft.com/visualstudio/making-repeated-edits-easier-with-intellicode-suggestions/}{making repeated edits easier with IntelliCode suggestions}'', by Peter Groenewegen.

\plush{}

\pptSection[Tests]{Test Case Generation}

Ponicode was alive in 2022, now it's \st{dead} acquired by CircleCI.

\plush{}

\pptSection[Refactor]{Automated Refactoring}

\begin{multicols}{2}
\pptPic{.9}{codescene.png}
\par\columnbreak\par
I found this picture in the \href{https://codescene.com/engineering-blog/refactoring-recommendations}{CodeScene website}.
\end{multicols}

\plush{}

\pptSection[LLM]{Large Language Models (LLM)}

\begin{multicols}{2}
\pptPic{.9}{gpt-1.png}
\par\columnbreak\par
\pptPic{0.9}{gpt-2.png}
\end{multicols}

\plush{\pptChapter{What's Next?}}

\pptSection[Neural]{Neural Software Analysis}

\textbf{``Neural Software Analysis''} \\
By Michael Pradel, Satish Chandra \\
\emph{Communications of the ACM}, January 2022, Vol. 65 No. 1, Pages 86--96

\href{https://www.youtube.com/watch?v=Q0XJ5DoplRU}{Watch it}.

\plush{}

\pptSection[EEG]{Electroencephalography (EEG)}

\textbf{``Understanding Programming Expertise: An Empirical Study of Phasic Brain Wave Changes''} \\
By Igor Crk, Timothy Kluthe, Andreas Stefik \\
\emph{ACM Transactions on Computer-Human Interaction}, Volume 23, Issue 1, Article No. 2, Pages 1--29

\plush{}

\end{document}
