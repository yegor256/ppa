% (The MIT License)
%
% Copyright (c) 2022-2023 Yegor Bugayenko
%
% Permission is hereby granted, free of charge, to any person obtaining a copy
% of this software and associated documentation files (the 'Software'), to deal
% in the Software without restriction, including without limitation the rights
% to use, copy, modify, merge, publish, distribute, sublicense, and/or sell
% copies of the Software, and to permit persons to whom the Software is
% furnished to do so, subject to the following conditions:
%
% The above copyright notice and this permission notice shall be included in all
% copies or substantial portions of the Software.
%
% THE SOFTWARE IS PROVIDED 'AS IS', WITHOUT WARRANTY OF ANY KIND, EXPRESS OR
% IMPLIED, INCLUDING BUT NOT LIMITED TO THE WARRANTIES OF MERCHANTABILITY,
% FITNESS FOR A PARTICULAR PURPOSE AND NONINFRINGEMENT. IN NO EVENT SHALL THE
% AUTHORS OR COPYRIGHT HOLDERS BE LIABLE FOR ANY CLAIM, DAMAGES OR OTHER
% LIABILITY, WHETHER IN AN ACTION OF CONTRACT, TORT OR OTHERWISE, ARISING FROM,
% OUT OF OR IN CONNECTION WITH THE SOFTWARE OR THE USE OR OTHER DEALINGS IN THE
% SOFTWARE.

\documentclass{article}
\usepackage{../ppa}
\newcommand*\thetitle{Model Checking}
\newcommand*\thesubtitle{...}
\begin{document}

\plush{\ppaTitlePage{9}}

% \emph{Model checking} is a method for checking whether a finite-state model of a system meets a given specification.

\pptToc

\plush{\pptChapter{Motivating Example}}

\pptSection{Div by Zero}

\begin{pptWide}{4}
{\small\begin{verbatim}
// Process no. 1:

extern int x;
extern double y;
int measure() {
  if (x != 0) {
    y = 1.0 / x;
  }
}
\end{verbatim}
}
\par\columnbreak\par
\scalebox{.8}{\begin{tikzpicture}[graph]
\node[draw=none,minimum width=0pt,inner sep=0pt] (start) {$\circ$};
\node[terminal,below=1cm of start] (neq) {\texttt{x != 0}}; \path (start) edge (neq);
\node[terminal,below right=1cm and .1cm of neq] (div) {\texttt{y = 1 / x}}; \path (neq) edge (div);
\node[draw=none,minimum width=0pt,inner sep=0pt,below=3cm of neq] (exit) {$\circ$};
\path (neq) edge (exit);
\path (exit) edge [color=orange, line width=2pt, bend left=90] (start);
\end{tikzpicture}}
\par\columnbreak\par
{\small\begin{verbatim}
// Process no. 2:

extern int x;
void roll() {
  x += 1;
  if (x > 10) {
    x -= 10;
  }
}
\end{verbatim}
}
\par\columnbreak\par
\scalebox{.8}{\begin{tikzpicture}[graph]
\node[draw=none,minimum width=0pt,inner sep=0pt] (start) {$\circ$};
\node[terminal,below=1cm of start] (plus) {\texttt{x = x + 1}}; \path (start) edge (plus);
\node[terminal,below=1cm of plus] (gt) {\texttt{x > 10}}; \path (plus) edge (gt);
\node[terminal,below right=1cm and .1cm of gt] (set) {\texttt{x = x - 10}}; \path (gt) edge (set);
\node[draw=none,minimum width=0pt,inner sep=0pt,below=3cm of gt] (exit) {$\circ$};
\path (gt) edge (exit);
\path (exit) edge [color=orange, line width=2pt, bend left=90] (start);
\end{tikzpicture}}
\end{pptWide}

Can we detect ``division by zero'' using symbolic execution? Is ``division by zero'' the only error here?

\plush{}

\pptSection[ProMeLa]{ProMeLa (Process Meta Language)}

\begin{pptWide}{2}
{\small\begin{verbatim}
extern int x;
extern double y;
int measure() {
  if (x != 0) {
    y = 1.0 / x;
  }
}
void roll() {
  x += 1;
  if (x > 10) {
    x -= 10;
  }
}
\end{verbatim}
}
\par\columnbreak\par
{\scriptsize\begin{verbatim}
int x; bool dbz;
active proctype measure() {
  do :: true ->
    if
    :: (x != 0) -> dbz = (x == 0)
    :: skip
    fi
  od
}
active proctype roll() {
  do :: true ->
    x = x + 1;
    if
    :: x > 10 -> x = x - 10
    :: skip
    fi
  od
}
\end{verbatim}
}
\end{pptWide}

\plush{}

\pptSection[SPIN]{SPIN (Simple ProMeLa Interpreter)}

\begin{pptWide}{2}
{\scriptsize\begin{verbatim}
int x; bool dbz;
active proctype measure() {
  do :: true ->
    if
    :: (x != 0) -> dbz = (x == 0)
    :: skip
    fi
  od
}
active proctype roll() {
  do :: true ->
    x = x + 1;
    if
    :: x > 10 -> x = x - 10
    :: skip
    fi;
    printf("x = %d\n", x);
  od
}
\end{verbatim}
}
\par\columnbreak\par
{\scriptsize\begin{verbatim}
$ spin main.pml | head
          x = 1
          x = 2
          x = 3
          x = 4
          x = 5
          x = 6
          x = 7
          x = 8
          x = 9
          x = 10
$ spin main.pml | tail
...
\end{verbatim}
}

Just checkout \href{https://github.com/nimble-code/Spin}{this repo} and run \texttt{make},
the \texttt{spin} binary will be compiled.
\end{pptWide}

\plush{}

\pptSection[Monitor]{Monitoring Process}

\begin{pptWide}{2}
{\scriptsize\begin{verbatim}
int x; bool dbz;
active proctype measure() {
  do :: true ->
    if
    :: (x != 0) -> dbz = (x == 0)
    :: skip
    fi
  od
}
active[2] proctype roll() {
  do :: true ->
    x = x + 1;
    if
    :: x > 10 -> x = x - 10
    :: skip
    fi
  od
}
\end{verbatim}
}
\par\columnbreak\par
{\scriptsize\begin{verbatim}
active proctype monitor() {
  do :: true ->
    assert(!dbz);
    assert(x >= 0);
  od
}
\end{verbatim}
}

Pay attention to the \texttt{[2]} suffix after the \texttt{active}
keyword. It tells SPIN to start two instances of the \texttt{roll} process.
\end{pptWide}

\plush{}

\pptSection[Assertion]{Fail on Assertion}

\begin{pptWide}{2}
{\scriptsize\begin{verbatim}
int x; bool dbz;
active proctype measure() {
  do :: true ->
    if
    :: (x != 0) -> dbz = (x == 0)
    :: skip
    fi
  od }
active[2] proctype roll() {
  do :: true ->
    x = x + 1;
    if
    :: x > 10 -> x = x - 10
    :: skip
    fi
  od }
active proctype monitor() {
  do :: true -> assert(!dbz); assert(x >= 0); od
}
\end{verbatim}
}
\par\columnbreak\par
{\scriptsize\begin{verbatim}
$ spin main.pml
spin: main.pml:22, Error: assertion violated
spin: text of failed assertion: assert((x>=0))
#processes: 4
    x = -9
    dbz = 0
584:  proc  3 (monitor:1) main.pml:22 (state 3)
584:  proc  2 (roll:1) main.pml:17 (state 7)
584:  proc  1 (roll:1) main.pml:18 (state 9)
584:  proc  0 (measure:1) main.pml:9 (state 8)
4 processes created
\end{verbatim}
}
\end{pptWide}

\plush{}

\plush{\pptChapter{Model-less Checking}}

\pptSection[JPF]{Java PathFinder}

...

\pptSection[Literature]{Further Reading/Watching}

Introduction lecture by \href{https://www.youtube.com/watch?v=VHWEldcSx14}{Joost-Pieter Katoen}

\href{https://spinroot.com/spin/Doc/p40-ben-ari.pdf}{A Primer on Model Checking} by Mordechai Ben-Ari

\end{document}
