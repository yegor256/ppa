% (The MIT License)
%
% Copyright (c) 2022 Yegor Bugayenko
%
% Permission is hereby granted, free of charge, to any person obtaining a copy
% of this software and associated documentation files (the 'Software'), to deal
% in the Software without restriction, including without limitation the rights
% to use, copy, modify, merge, publish, distribute, sublicense, and/or sell
% copies of the Software, and to permit persons to whom the Software is
% furnished to do so, subject to the following conditions:
%
% The above copyright notice and this permission notice shall be included in all
% copies or substantial portions of the Software.
%
% THE SOFTWARE IS PROVIDED 'AS IS', WITHOUT WARRANTY OF ANY KIND, EXPRESS OR
% IMPLIED, INCLUDING BUT NOT LIMITED TO THE WARRANTIES OF MERCHANTABILITY,
% FITNESS FOR A PARTICULAR PURPOSE AND NONINFRINGEMENT. IN NO EVENT SHALL THE
% AUTHORS OR COPYRIGHT HOLDERS BE LIABLE FOR ANY CLAIM, DAMAGES OR OTHER
% LIABILITY, WHETHER IN AN ACTION OF CONTRACT, TORT OR OTHERWISE, ARISING FROM,
% OUT OF OR IN CONNECTION WITH THE SOFTWARE OR THE USE OR OTHER DEALINGS IN THE
% SOFTWARE.

\documentclass[nobrand,anonymous,nodate,nosecurity]{huawei}
\usepackage{multicol}
\usepackage{href-ul}
\usepackage{ffcode}
\begin{document}

{\sffamily{\bfseries\Large Practical Program Analysis}\\
Series of lectures by \href{https://www.yegor256.com}{Yegor Bugayenko}
% to students of \href{https://innopolis.university/en/}{Innopolis University} in 2021,\\
% and \href{https://www.youtube.com/playlist?list=PLaIsQH4uc08woJKRAA7mmjs9fU0jeKjjM}{video recorded}}

% The entire set of slide decks is in \href{https://github.com/yegor256/ssd16}{yegor256/ssd16} GitHub repository.

\begin{abstract}
The course is a high-level introduction to program analysis with a
strong emphasis on its practical implementation in programming language
design and static analyzers. Students may listen to this
course if they plan to develop their own programming languages,
compilers, IDEs, static analyzers, code refactoring, generating and optimization tools.
The course combines theoretical study with the development of
instruments that analyze source code and automatically modifies it.
\end{abstract}

% \section*{Introduction}

\textbf{What is the goal?}\\
The primary objective of the course is to demonstrate
how theoretical knowledge of program analysis may be applied to
the design of software tools.

\textbf{Who is the teacher?}\\
I'm developing software for more than 30 years, being a hands-on programmer
(see my GitHub account: \href{https://github.com/yegor256}{@yegor256})
and a manager of other programmers. At the moment I'm a director
of an R\&D laboratory in Huawei. Our primary research focus is
software quality problems. You may find some lectures I've presented
at some software conferences on
\href{https://www.youtube.com/channel/UCr9qCdqXLm2SU0BIs6d_68Q}{my YouTube channel}.
I also published \href{https://www.yegor256.com/books.html}{a few books}
and wrote \href{https://www.yegor256.com/contents.html}{a blog} about software engineering
and object-oriented programming.
I previously tought two courses in
Innopolis University (Kazan, Russia)
and HSE University (Moscow, Russia):
\href{https://github.com/yegor256/ssd16}{Software Systems Design}
and
\href{https://github.com/yegor256/eqsp}{Ensuring Quality in Software Projects}
(all videos are available).

\textbf{Why this course?}\\
The quality of software code that most of us programmers write is way below
the expectations of our customers. Two main reasons for that
is the lack of understanding of how programming languages are designed
internally and the absences of connection between theoretical knowledge
about language design and the actual software we use every day to
write code: IDEs, compilers, code analyzers and modifiers. This course
may help build the bridge between theory and practice.

\textbf{What's the methodology?}\\
The course is organized in pairs of lectures. The first lecture in a pair
is an introduction of a theory, while the second lecture is a demonstration
of how the theory may be applied to the development of a software tool. During
the course students will participate in the development of an automated
code optimization tool for one of popular programming languages, like Java or C++.

\newpage
\section*{Course Structure}

Prerequisites to the course (it is expected that a student knows this):

\begin{itemize}
\item How to write code
\item How to design software
\end{itemize}

After the course a student \emph{hopefully} will understand the basics of:

\begin{itemize}
\item Abstract Interpretation
\item Symbolic Execution
\item Model Checking
\item Type theory
\item $\lambda$-calculus
\item Functional Programming
\item Axiomatic Semantics
\item Operational Semantics
\item Temporal Logic
\end{itemize}

Also, a student will be able to develop:

\begin{itemize}
\item A Programming Language
\item A Compiler
\item A Static Analyzer
\item A Code Refactoring Tool
\end{itemize}

% \newpage
% \section*{Lectures}

% \newcommand\github[1]{\href{https://github.com/#1}{#1}}

% This is a list of cases that will be discussed at the lectures:

% \begin{enumerate}
% \end{enumerate}

% Students are welcome to pick most interesting cases studies.

% \newpage
% \section*{Laboratory Classes}

% A few following laboratory classes may support the course, where students
% will be asked to solve some of these tasks (the most complex are at the bottom):

% \begin{enumerate}
% \end{enumerate}

% There could be other tasks too.

\newpage
\section*{Grading}

Students may form groups of up to four people. Each group will present
their own public GitHub repository with a software module inside. The group
will make a presentation of the moduld that is
present in the repository. They will have to explain during a 10-minutes
oral presentation with live GitHub demonstration via screen sharing:

\begin{itemize}
	\item How the software is designed?
	\item What theory is used in it?
	\item How far away it is from the canonical theory?
	\item How the result are validated (tested)?
\end{itemize}

Most probably, there will
be no more than 20\% of ``A'' marks, no more than 40\% of ``B,''
and the rest will go to ``C'' and ``D.'' However, this distribution is
not mandatory: if all students make excellent presentations, everybody
will get ``A.''

Attendance will be tracked at the lectures. If you attend more than 75\%
of all lectures, you will not get less than ``C''.

% At the laboratory classes each group will have to complete three
% home works and defend them verbally on-site.
% A completion of less than two will give everybody in the group a negative point,
% a completion of three --- will give a positive point; the point will be added
% to the grade given by the lecturer.

% Higher grades will be given for:

% \begin{itemize}
% 	\item Better understanding of the reasons behind used mechanisms,
% 	\item How they help ensure quality,
% 	\item How often they get activated,
% 	and
% 	\item What are the drawbacks of them.
% \end{itemize}

A retake exam is possible, following exactly the same procedure. However,
the highest mark most probably possible at the retake is ``C.''

% Students are highly advised to discuss
% their repositories and quality ensuring mechanisms with each other,
% before the final exam, in order to understand their relative positions
% and maybe trigger new ideas.

% \newpage
% \section*{Learning Material}

% The following books are highly recommended to read (in no particular order):

% \begin{multicols}{2}\small\raggedright
% {\nospell{Steve McConnell}}, \emph{Software Estimation: Demystifying the Black Art}\\[3pt]
% {Robert Martin}, \emph{Clean Architecture: A Craftsman's Guide to Software Structure and Design}\\[3pt]
% {Steve McConnell}, \emph{Code Complete}\\[3pt]
% {Frederick Brooks Jr.}, \emph{Mythical Man-Month, The: Essays on Software Engineering}\\[3pt]
% {David Thomas et al.}, \emph{The Pragmatic Programmer: Your Journey To Mastery}\\[3pt]
% {Robert C. Martin}, \emph{Clean Code: A Handbook of Agile Software Craftsmanship}\\[3pt]
% {David West}, \emph{Object Thinking}\\[3pt]
% {Yegor Bugayenko}, \emph{Code Ahead}\\[3pt]
% {Michael Feathers}, \emph{Working Effectively with Legacy Code}\\[3pt]
% {\nospell{Jez Humble} et al.}, \emph{Continuous Delivery: Reliable Software Releases through Build, Test, and Deployment Automation}\\[3pt]
% {\nospell{Michael T. Nygard}}, \emph{Release It!: Design and Deploy Production-Ready Software}\\[3pt]
% \end{multicols}

\end{document}
