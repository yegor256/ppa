% (The MIT License)
%
% Copyright (c) 2022-2024 Yegor Bugayenko
%
% Permission is hereby granted, free of charge, to any person obtaining a copy
% of this software and associated documentation files (the 'Software'), to deal
% in the Software without restriction, including without limitation the rights
% to use, copy, modify, merge, publish, distribute, sublicense, and/or sell
% copies of the Software, and to permit persons to whom the Software is
% furnished to do so, subject to the following conditions:
%
% The above copyright notice and this permission notice shall be included in all
% copies or substantial portions of the Software.
%
% THE SOFTWARE IS PROVIDED 'AS IS', WITHOUT WARRANTY OF ANY KIND, EXPRESS OR
% IMPLIED, INCLUDING BUT NOT LIMITED TO THE WARRANTIES OF MERCHANTABILITY,
% FITNESS FOR A PARTICULAR PURPOSE AND NONINFRINGEMENT. IN NO EVENT SHALL THE
% AUTHORS OR COPYRIGHT HOLDERS BE LIABLE FOR ANY CLAIM, DAMAGES OR OTHER
% LIABILITY, WHETHER IN AN ACTION OF CONTRACT, TORT OR OTHERWISE, ARISING FROM,
% OUT OF OR IN CONNECTION WITH THE SOFTWARE OR THE USE OR OTHER DEALINGS IN THE
% SOFTWARE.

\documentclass{article}
\usepackage{../ppa}
\usepackage{booktabs}
\newcommand*\thetitle{Data Flow Analysis}
\newcommand*\thesubtitle{Example, Method, Lattice, Precision, Sensitivity}
\begin{document}

\plush{\lnTitlePage{7}{10}{loAVqyfP374}}

\pptToc

\plush{\pptChapter[Example]{Motivating Example}}

\pptBanner{Unassigned Variable}

Which code snippet has an error (``use of unassigned variable'')?

\begin{multicols}{2}
{\ttfamily
int f(int x) \{ \\
\quad  int a; \\
\quad  if (x > 10) \\
\quad\quad a = 42; \\
\quad  while (x++ < 5) \\
\quad\quad    a = x; \\
\quad  return a + 1; \\
\}}
\par\columnbreak\par
{\ttfamily
int f(int x) \{ \\
\quad  int a; \\
\quad  if (x > 3) \\
\quad\quad a = 42; \\
\quad  while (x++ < 12) \\
\quad\quad    a = x; \\
\quad  return a + 1; \\
\}}
\end{multicols}

\plush{}

\pptBanner{ChatGPT vs. Clang Tidy}

\begin{multicols}{2}
\pptPic{0.85}{gpt.png}
\par\columnbreak\par
\pptPic{1}{tidy.png}
\end{multicols}

\plush{}

\pptBanner{UndefinedBehaviorSanitizer}

Clang \href{https://clang.llvm.org/docs/UndefinedBehaviorSanitizer.html}{UndefinedBehaviorSanitizer} (the dynamic analyzer) can detect the issue in runtime:

\pptPic{0.8}{clang.png}

\plush{}

\pptBanner{IntelliJ IDE}

IntelliJ IDEA doesn't see the difference:

\pptPic{0.8}{intellij.png}

\plush{}

\pptBanner{Java Compiler}

\texttt{javac} doesn't see the difference either:

\pptPic{0.8}{javac.png}

\plush{}

\plush{\pptChapter{Method}}

\pptSection[CFG]{Control Flow Graph}

First, we represent the program as a Control Flow Graph (CFG):

\begin{multicols}{2}
{\ttfamily
int f(int x) \{ \\
\quad  int a; \\
\quad  if (x > 10) \\
\quad\quad a = 42; \\
\quad  while (x++ < 5) \\
\quad\quad    a = x; \\
\quad  return a + 1; \\
\}}
\par\columnbreak\par
\scalebox{.9}{\begin{tikzpicture}[graph]
\node[draw=none,minimum width=0pt,inner sep=0pt] (start) {$\circ$};
\node[terminal,below=1cm of start] (gt) {x > 10}; \path (start) edge (gt);
\node[terminal,below right=1cm of gt] (42) {a = 42}; \path (gt) edge (42);
\node[terminal,below left=1cm of 42] (while) {x++ < 5}; \path (42) edge (while); \path (gt) edge [bend right=15] (while);
\node[terminal,below=1cm of while] (x) {a = x}; \path (while) edge (x);
\node[terminal,below=1cm of x] (plus) {a + 1};
\path (x.east) edge [bend right=90] (while.east);
\path (while.west) edge [bend right=90] (plus.west);
\end{tikzpicture}}
\end{multicols}

\plush{}

\pptSection[Properties]{Six Properties of Data Flow Analysis}

Data flow analysis \emph{propagates} information (\emph{data}) along the control flow graph, with the following six properties in mind:

\begin{enumerate}
    \item Domain (of \emph{data flow facts})
    \item Direction (forward or backward)
    \item \emph{Transfer Function} (sometimes with GEN and KILL sets)
    \item \emph{Confluence Operator} ("meet" or "join")
    \item Boundary Condition (start data for the entry node)
    \item Initial Values (start data for each node)
\end{enumerate}

\plush{}

\pptSection{Over-approximation}

\begin{multicols}{2}
\scalebox{.9}{\begin{tikzpicture}[graph]
\node[draw=none,minimum width=0pt,inner sep=0pt] (start) {$\circ$};
\node[terminal,below=1cm of start] (gt) {x > 10}; \path (start) edge (gt);
\node[terminal,below right=1cm of gt] (42) {a := 42}; \path (gt) edge (42);
\node[terminal,below left=1cm of 42] (while) {x++ < 5}; \path (42) edge (while); \path (gt) edge [bend right=15] (while);
\node[terminal,below=1cm of while] (x) {a := x}; \path (while) edge (x);
\node[terminal,below=1cm of x] (plus) {a + 1};
\path (x.east) edge [bend right=90] (while.east);
\path (while.west) edge [bend right=90] (plus.west);
\end{tikzpicture}}
\par\columnbreak\par
\begin{enumerate}\small\setlength\itemsep{-3pt}
    \item Domain:\\
    variable names
    \item Direction: \\
    forward
    \item Transfer Function: \\
    add on ``:=''
    \item Confluence Operator:\\
    meet, intersection
    \item Boundary Condition:\\
    \(\{x\}\)
    \item Initial Values:\\
    empty sets
\end{enumerate}
\end{multicols}

\plush{}

\pptSection[Meet]{Meet Operator}

The \emph{meet operator} is coming from the lattice that abstracts the data that flows (remember \emph{abstract interpretation}?):

\begin{multicols}{2}
\scalebox{.8}{\begin{tikzpicture}[graph]
\node[draw=none,minimum width=0pt,inner sep=0pt] (start) {$\circ$};
\node[terminal,below=1cm of start] (gt) {x > 10}; \path (start) edge (gt);
\node[terminal,below right=1cm of gt] (42) {a := 42}; \path (gt) edge (42);
\node[terminal,below left=1cm of 42] (while) {x++ < 5}; \path (42) edge (while); \path (gt) edge [bend right=15] (while);
\node[terminal,below=1cm of while] (x) {a := x}; \path (while) edge (x);
\node[terminal,below=1cm of x] (plus) {a + 1};
\path (x.east) edge [bend right=90] (while.east);
\path (while.west) edge [bend right=90] (plus.west);
\end{tikzpicture}}
\par\columnbreak\par
\scalebox{.8}{\begin{tikzpicture}[line width=1pt,->]
\node (top) {\(\{x, a\}\)};
\node [below left=1cm of top] (x) {\(\{x\}\)};
\node [below right=1cm of top] (a) {\(\{a\}\)};
\node [below=3cm of top] (bot) {\(\{\}\)};
\draw (x) -- (top);
\draw (a) -- (top);
\draw (bot) -- (x);
\draw (bot) -- (a);
\end{tikzpicture}}

\( \{x, a\} \sqcap \{x\} \to \dots \)

\( \{a\} \sqcap \{\} \to \dots \)

\( \{a\} \sqcap \{x\} \to \dots \)
\end{multicols}

\plush{}

\pptSection[GEN/KILL]{GEN and KILL Functions}

A transfer function may be defined by defining \texttt{GEN} and \texttt{KILL} functions:

\begin{multicols}{2}
\scalebox{.8}{\begin{tikzpicture}[graph]
\node[draw=none,minimum width=0pt,inner sep=0pt] (start) {$\circ$};
\node[terminal,below=1cm of start] (gt) {x > 10}; \path (start) edge (gt);
\node[terminal,below right=1cm of gt] (42) {a := 42}; \path (gt) edge (42);
\node[terminal,below left=1cm of 42] (while) {x++ < 5}; \path (42) edge (while); \path (gt) edge [bend right=15] (while);
\node[terminal,below=1cm of while] (x) {a := x}; \path (while) edge (x);
\node[terminal,below=1cm of x] (plus) {a + 1};
\path (x.east) edge [bend right=90] (while.east);
\path (while.west) edge [bend right=90] (plus.west);
\end{tikzpicture}}
\par\columnbreak\par
\begin{tabularx}{\columnwidth}{Xll}
\toprule
\(s\) & \(\texttt{GEN}(s)\) & \(\texttt{KILL}(s)\) \\
\midrule
\texttt{x > 10} & \(\{\}\) & \(\{\}\) \\
\texttt{a := 42} & \(\{a\}\) & \(\{\}\) \\
\texttt{x++ < 5} & \(\{\}\) & \(\{\}\) \\
\texttt{a := x} & \(\{a\}\) & \(\{\}\) \\
\texttt{a + 1} & \(\{\}\) & \(\{\}\) \\
\bottomrule
\end{tabularx}
\end{multicols}

\plush{}

\pptSection[Low]{Over-approximation = Low Precision}

From the perspective of \emph{path insensitive} data flow analysis, there are bugs in both CFGs, but it's wrong:

\begin{multicols}{3}
\scalebox{.8}{\begin{tikzpicture}[graph]
\node[draw=none,minimum width=0pt,inner sep=0pt] (start) {$\circ$};
\node[terminal,below=1cm of start] (gt) {x > 10}; \path (start) edge (gt);
\node[terminal,below right=1cm of gt] (42) {a := 42}; \path (gt) edge (42);
\node[terminal,below left=1cm of 42] (while) {x++ < 5}; \path (42) edge (while); \path (gt) edge [bend right=15] (while);
\node[terminal,below=1cm of while] (x) {a := x}; \path (while) edge (x);
\node[terminal,below=1cm of x] (plus) {a + 1};
\path (x.east) edge [bend right=90] (while.east);
\path (while.west) edge [bend right=90] (plus.west);
\end{tikzpicture}}
\par\columnbreak\par
\scalebox{.8}{\begin{tikzpicture}[graph]
\node[draw=none,minimum width=0pt,inner sep=0pt] (start) {$\circ$};
\node[terminal,below=1cm of start] (gt) {x > 3}; \path (start) edge (gt);
\node[terminal,below right=1cm of gt] (42) {a := 42}; \path (gt) edge (42);
\node[terminal,below left=1cm of 42] (while) {x++ < 12}; \path (42) edge (while); \path (gt) edge [bend right=15] (while);
\node[terminal,below=1cm of while] (x) {a := x}; \path (while) edge (x);
\node[terminal,below=1cm of x] (plus) {a + 1};
\path (x.east) edge [bend right=90] (while.east);
\path (while.west) edge [bend right=90] (plus.west);
\end{tikzpicture}}
\par\columnbreak\par
\scalebox{.8}{\begin{tikzpicture}[line width=1pt,->]
\node (top) {\(\{x, a\}\)};
\node [below left=1cm of top] (x) {\(\{x\}\)};
\node [below right=1cm of top] (a) {\(\{a\}\)};
\node [below=3cm of top] (bot) {\(\{\}\)};
\draw (x) -- (top);
\draw (a) -- (top);
\draw (bot) -- (x);
\draw (bot) -- (a);
\end{tikzpicture}}
\end{multicols}

\plush{}

\plush{\pptChapter{Sensitivities}}

\pptSection{Path-Sensitive Analysis}

A \emph{path-sensitive} analysis computes different pieces of analysis information dependent on the \emph{predicates} at conditional branch instructions.

\scalebox{.9}{\begin{tikzpicture}[graph]
\node[draw=none,minimum width=0pt,inner sep=0pt] (start) {$\circ$};
\node[terminal,below=1cm of start] (gt) {x > 3}; \path (start) edge (gt);
\node[terminal,below right=1cm of gt] (42) {a := 42}; \path (gt) edge (42);
\node[terminal,below left=1cm of 42] (while) {x++ < 12}; \path (42) edge (while); \path (gt) edge [bend right=15] (while);
\node[terminal,below=1cm of while] (x) {a := x}; \path (while) edge (x);
\node[terminal,below=1cm of x] (plus) {a + 1};
\path (x.east) edge [bend right=90] (while.east);
\path (while.west) edge [bend right=90] (plus.west);
\end{tikzpicture}}

\plush{}

\pptSection{Flow-Sensitive Analysis}

A \emph{flow-sensitive} analysis takes into account the order of statements in a program.

The analysis we did before was flow sensitive. Flow \emph{insensitive} analysis example:

{\small
\begin{ffcode}
a = 0;
a = 5;
a = a + 1;
// What is a possible value of 'a'?
\end{ffcode}
}

\plush{}

\pptSection{Context-Sensitive Analysis}

A \emph{context-sensitive} analysis is an interprocedural analysis that considers the calling context when analyzing the target of a function call.

{\small
\begin{ffcode}
f(5, 6); // call-site #1
f(6, 5); // call-site #2
void f(x, y) {
  // Is it possible to have x == y?
}
\end{ffcode}
}

\plush{}

\plush{\pptChapter[Types]{Most Common Types}}

\pptSection[Reaching]{Reaching Definitions Analysis}

\emph{Reaching definitions} is a data-flow analysis which statically determines which definitions may reach a given point in the code.

{\small
\begin{ffcode}
float price(int book) {
  float p = load_from_database();
  if (book < 100)
    p = 14.99;
  if (book > 50)
    p = 9.99;
  float discount = 0.90;
  return p * discount;
}
\end{ffcode}
}

Do you see any problems with this code?

\plush{}

\pptSection[Liveness]{Liveness Analysis}

\emph{Live variable analysis} calculates the variables that are live at each point in the program (they hold values that may be needed in the future).

{\small
\begin{ffcode}
int price(int book_id) {
  int p;
  int discount;
  if (book_id > 400)
    discount = 10;
  p = load_price_from_database(book_id);
  p = ( p * 95 ) / 100;
  return p;
}
\end{ffcode}
}

Do you see any problems in the code?

\plush{}

\pptSection[Assignment]{Definite Assignment Analysis}

\emph{Definite assignment analysis} conservatively ensures that a variable or location is always assigned before it is used.

{\small
\begin{ffcode}
int salary(int user_id) {
  int s;
  if (user_id > 400) {
    s = get_salary_from_mysql(user_id);
  } else if (user_id < 400) {
    s = 0;
  }
  return s;
}
\end{ffcode}
}

Is there an error in this code?

\plush{}

\pptSection[Available]{Available Expression Analysis}

\emph{Available expression analysis} determines for each point in the program the set of expressions that need not be recomputed.

{\small
\begin{ffcode}
int price(int book_id) {
  int p = 14;
  if (stock(book_id) < 100) {
    p = 19;
  } else if (stock(book_id) > 1000) {
    p = 9;
  }
  return p;
}
\end{ffcode}
}

Shall we computer \texttt{stock(book\_id)} twice?

\plush{}

\pptSection[Constants]{Constant Propagation Analysis}

\emph{Constant propagation analysis} at every statement tells which variables is a constant:
every execution that reaches that point, gives that variable the same value.

{\small
\begin{ffcode}
float discount(float price) {
  float d = 0.8;
  if (price < 14.99)
    d = 0.93;
  else
    d = d + 0.13;
  return price * d;
}
\end{ffcode}
}

Is there an error in this code?

\plush{}

\plush{\pptChapter[Literature]{Further Reading/Watching}}

Book and \href{http://cs.au.dk/~amoeller/spa/}{slides} by Anders M\o{}ller et al.

Lectures of Michael Pradel on \href{https://www.youtube.com/watch?v=rJYgTJaXZU0}{YouTube}.

\end{document}
