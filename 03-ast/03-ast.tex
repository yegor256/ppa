% (The MIT License)
%
% Copyright (c) 2022-2023 Yegor Bugayenko
%
% Permission is hereby granted, free of charge, to any person obtaining a copy
% of this software and associated documentation files (the 'Software'), to deal
% in the Software without restriction, including without limitation the rights
% to use, copy, modify, merge, publish, distribute, sublicense, and/or sell
% copies of the Software, and to permit persons to whom the Software is
% furnished to do so, subject to the following conditions:
%
% The above copyright notice and this permission notice shall be included in all
% copies or substantial portions of the Software.
%
% THE SOFTWARE IS PROVIDED 'AS IS', WITHOUT WARRANTY OF ANY KIND, EXPRESS OR
% IMPLIED, INCLUDING BUT NOT LIMITED TO THE WARRANTIES OF MERCHANTABILITY,
% FITNESS FOR A PARTICULAR PURPOSE AND NONINFRINGEMENT. IN NO EVENT SHALL THE
% AUTHORS OR COPYRIGHT HOLDERS BE LIABLE FOR ANY CLAIM, DAMAGES OR OTHER
% LIABILITY, WHETHER IN AN ACTION OF CONTRACT, TORT OR OTHERWISE, ARISING FROM,
% OUT OF OR IN CONNECTION WITH THE SOFTWARE OR THE USE OR OTHER DEALINGS IN THE
% SOFTWARE.

\documentclass{article}
\usepackage{../ppa}
\newcommand*\thetitle{Abstract Syntax Tree}
\newcommand*\thesubtitle{...}
\begin{document}

\plush{\innoTitlePage{3}}

\pptToc

\plush{\pptChapter[Trees]{Concrete vs. Abstract}}

The \emph{concrete syntax} of a programming language is defined by a context free grammar (CFG).
The \emph{abstract syntax} of an implementation is the set of trees used to represent programs in the implementation.
\plush{}

Simple program:
\begin{ffcode}
PRINT "Hi," + name;
EXIT;
\end{ffcode}

\begin{multicols}{2}
Concrete Syntax Tree:

\scalebox{.6}{\begin{tikzpicture}[graph]
\node (p) {$P$};
\node[below left=1cm and 4cm of p] (l1) {$L$}; \path (p) edge (l1);
\node[below left=1cm and 2cm of l1] (print) {$C$}; \path (l1) edge (print);
\node[terminal,below=1cm of print] (print-t) {\texttt{PRINT}}; \path (print) edge (print-t);
\node[below left=1cm and 0cm of l1] (hello) {$S$}; \path (l1) edge (hello);
\node[terminal,below=3cm of hello] (hello-t) {\texttt{"Hi,"}}; \path (hello) edge (hello-t);
\node[below right=1cm and 0cm of l1] (plus) {$O$}; \path (l1) edge (plus);
\node[terminal,below=1cm of plus] (plus-t) {\texttt{+}}; \path (plus) edge (plus-t);
\node[below right=1cm and 2cm of l1] (name) {$V$}; \path (l1) edge (name);
\node[terminal,below=3cm of name] (name-t) {\texttt{name}}; \path (name) edge (name-t);
\node[terminal,below right=1cm and 5cm of l1] (semi1) {\texttt{";"}}; \path (l1) edge (semi1);
%
\node[below right=1cm and 6cm of p] (l2) {$L$}; \path (p) edge (l2);
\node[below left=1cm of l2] (exit) {$C$}; \path (l2) edge (exit);
\node[terminal,below=1cm of exit] (exit-t) {\texttt{EXIT}}; \path (exit) edge (exit-t);
\node[terminal,below right=1cm of l2] (semi2) {\texttt{";"}}; \path (l2) edge (semi2);
\end{tikzpicture}}
\par\columnbreak\par
Abstract Syntax Tree:

\scalebox{.75}{\begin{tikzpicture}[graph]
\node[terminal] (p) {program};
\node[terminal,below left=1cm of p] (print) {print}; \path (p) edge (print);
\node[terminal,below right=1cm of p] (exit) {exit}; \path (p) edge (exit);
\node[terminal,below=1cm of print] (plus) {plus}; \path (print) edge (plus);
\node[terminal,below left=1cm of plus] (a1) {"Hi,"}; \path (plus) edge (a1);
\node[terminal,below right=1cm of plus] (a2) {name}; \path (plus) edge (a2);
\end{tikzpicture}}
\end{multicols}
\plush{}

\plush{\pptChapter[Contextual]{Contextual Analysis}}

\pptSection{Identification}

\begin{multicols}{2}
{\ttfamily
var x; \\
loop \{ var x; x++; \}; \\
print x;}
\par\columnbreak\par
\begin{tikzpicture}[graph]
\node[terminal] (p) {program};
\node[terminal,below left=1cm of p] (var1) {var x}; \path (p) edge (var1);
\node[terminal,below=1.2cm of p] (loop) {loop}; \path (p) edge (loop);
\node[terminal,below right=1cm and 3cm of p] (print) {print}; \path (p) edge (print);
\node[terminal,below=1cm of print] (x2) {x}; \path (print) edge (x2);
\node[terminal,below left=1cm of loop] (var2) {var x}; \path (loop) edge (var2);
\node[terminal,below right=1cm of loop] (plus) {++}; \path (loop) edge (plus);
\node[terminal,below=1cm of plus] (x1) {x}; \path (plus) edge (x1);
\end{tikzpicture}
\end{multicols}

Somehow we must \emph{link} different $x$ to different places, where they are \emph{declared},
maybe with the help of "\emph{ID table}," or by attaching attributes to AST nodes, or both.
\plush{}

\pptSection{Type Checking}



\pptSection{Type Inference}


\end{document}
