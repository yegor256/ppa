% (The MIT License)
%
% Copyright (c) 2022-2023 Yegor Bugayenko
%
% Permission is hereby granted, free of charge, to any person obtaining a copy
% of this software and associated documentation files (the 'Software'), to deal
% in the Software without restriction, including without limitation the rights
% to use, copy, modify, merge, publish, distribute, sublicense, and/or sell
% copies of the Software, and to permit persons to whom the Software is
% furnished to do so, subject to the following conditions:
%
% The above copyright notice and this permission notice shall be included in all
% copies or substantial portions of the Software.
%
% THE SOFTWARE IS PROVIDED 'AS IS', WITHOUT WARRANTY OF ANY KIND, EXPRESS OR
% IMPLIED, INCLUDING BUT NOT LIMITED TO THE WARRANTIES OF MERCHANTABILITY,
% FITNESS FOR A PARTICULAR PURPOSE AND NONINFRINGEMENT. IN NO EVENT SHALL THE
% AUTHORS OR COPYRIGHT HOLDERS BE LIABLE FOR ANY CLAIM, DAMAGES OR OTHER
% LIABILITY, WHETHER IN AN ACTION OF CONTRACT, TORT OR OTHERWISE, ARISING FROM,
% OUT OF OR IN CONNECTION WITH THE SOFTWARE OR THE USE OR OTHER DEALINGS IN THE
% SOFTWARE.

\documentclass[sigsoft,12pt,nonacm]{acmart}
\usepackage[T1]{fontenc}
\usepackage[utf8]{inputenc}
\usepackage{bm}
\usepackage{mathtools}
\usepackage{ffcode}
\usepackage{soul}
\renewcommand\emph[1]{\ul{#1}}
\title{Practical Program Analysis}
\begin{document}

\section{Formal Grammar}

\subsection{Notation}

Assume, we want to create a new programming language (very similar to Basic),
which will allow us to write programs that look like this:

\begin{ffcode}
10 PRINT "What is your name?"
20 INPUT X
30 PRINT "Hello,", X
\end{ffcode}

A \emph{grammar} $G$, according to Noam Chomsky (1956), is a tuple $\langle N, T, P, S\rangle$, where:
\begin{itemize}
  \item $N = \{ P_\texttt{rogram}, I_\texttt{nteger},
    N_\texttt{ame}, T_\texttt{ext}, L_\texttt{ine}, \dots \}$ (\emph{non-terminals} or \emph{variables})
  \item $T = \{ \texttt{10}, \texttt{20},
    \texttt{PRINT}, \texttt{X}, \texttt{,}, \texttt{"Hello"}, \dots \}$ (\emph{terminals} or \emph{alphabet})
  \item $P = \{ \dots \}$ (\emph{production rules})
  \item $S \in N$ (\emph{start symbol})
\end{itemize}

Sometimes it may be $\langle V, \Sigma, R, S\rangle$, but it's not important.

A \emph{language} that can be built by $G$ is denoted as $\bm{L}(G)$.

A \emph{production rule} specifies a replacement of its \emph{left-hand side} with its \emph{right-hand side}, for example:
\begin{equation*}
\begin{split}
p_1)\; & P_\texttt{rogram} \to P_\texttt{rogram} \; L_\texttt{ine} \\
p_2)\; & P_\texttt{rogram} \to \epsilon \\
p_3)\; & L_\texttt{ine} \to I_\texttt{nteger} \; C_\texttt{ommand} \; T_\texttt{ail} \\
p_4)\; & T_\texttt{ail} \to T_\texttt{ail} \; A_\texttt{rgument} \\
p_5)\; & T_\texttt{ail} \to \epsilon \\
p_6)\; & I_\texttt{nteger} \to \texttt{10} \\
p_7)\; & I_\texttt{nteger} \to \texttt{20} \\
p_8)\; & I_\texttt{nteger} \to \texttt{30} \\
\end{split}
\end{equation*}

Formally, a production rule is (using \emph{Kleene star}):
\begin{equation*}
(T \cup N)^* N (T \cup N)^* \to (T \cup N)^*
\end{equation*}

Each left-hand side must contain at least one nonterminal symbol.

Derivation process may be desribed using $\xRightarrow[p_i]{}$ notation:
\begin{equation*}
\begin{split}
P &\xRightarrow[p_1]{} \bm{P} \; \bm{L} \\
  &\xRightarrow[p_3]{} P \; \bm{I} \; \bm{C} \; \bm{T} \\
  &\xRightarrow[p_8]{} P \; \textbf{\texttt{30}} \; C \; T \\
  &\xRightarrow[p_?]{} P \; \texttt{30} \; \textbf{\texttt{PRINT}} \; T \\
  &\xRightarrow[p_1]{} \bm{P} \; \bm{L} \; \texttt{30} \; \texttt{PRINT} \; T \\
  &\xRightarrow[p_?]{} \dots \\
\end{split}
\end{equation*}

We can say that ``$G$ derives in zero or more steps'': $\xRightarrow[G]{*}$
(it is reflexive transitive closure of $\xRightarrow[G]{}$).

\subsection{Chomsky Hierarchy}

There are four types of grammars:
\begin{enumerate}
  \item[0] Unrestricted grammars
  \item[1] Context-sensitive grammars
  \item[2] Context-free grammars
  \item[3] Regular grammars
\end{enumerate}

\subsection{Context Free Grammar}

A \emph{context-free grammar} (CFG) is a grammar in which the left-hand side of each production rule consists of only a single nonterminal symbol.

Languages generated by context-free grammars are known as \emph{context-free languages} (CFL).

Not all languages can be generated by CFGs.

The \emph{language equality} question (do two given context-free grammars generate the same language?) is undecidable.
The \emph{language inclusion} question is also undecidable: Given two CFGs, can the first one generate all strings that the second one can generate?

The \emph{emptiness problem} (whether the grammar generates any terminal strings at all), is undecidable for context-sensitive grammars, but decidable for CFGs.

\subsection{Regular Grammar}

In a \emph{regular grammar} all production rules have at most one non-terminal symbol in the right part of the rule
($A$ and $B$ are non-terminals and $a$ is a string of terminals):
\begin{equation*}
\begin{split}
A &\to a \\
A &\to a \; B \quad \text{(\emph{right-linear grammar})} \\
A &\to B \; a \quad \text{(\emph{left-linear grammar})} \\
A &\to \epsilon \\
\end{split}
\end{equation*}

Left-linear grammar is just another name for left-regular grammar (the same for right-).

\subsection{Parse Tree}

A \emph{parse tree} (parsing tree, derivation tree, concrete syntax tree) is an ordered, rooted tree that represents the syntactic structure of a string according to some CFG.

\begin{tikzpicture}

\end{tikzpicture}

\subsection{Ambiguous Grammar}

An \emph{ambiguous grammar} is a CFG for which there exists a string that can have more than one leftmost derivation or parse tree.


\end{document}
