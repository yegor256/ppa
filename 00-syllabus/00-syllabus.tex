% (The MIT License)
%
% SPDX-FileCopyrightText: Copyright (c) 2022-2025 Yegor Bugayenko
% SPDX-License-Identifier: MIT

\documentclass[nobrand,anonymous,nodate,nosecurity]{huawei}
\usepackage{enumerate}
\usepackage{multicol}
\usepackage{href-ul}
\usepackage{ffcode}
\begin{document}

{\sffamily{\bfseries\Large Practical Program Analysis}\\
Series of lectures by \href{https://www.yegor256.com}{Yegor Bugayenko}
to students of \href{https://innopolis.university/en/}{Innopolis University} in 2023,\\
and \href{https://www.youtube.com/playlist?list=PLaIsQH4uc08wdXIC4utfgMxV_iswE9_Md}{video recorded}}

The entire set of slide decks is in \href{https://github.com/yegor256/ppa}{yegor256/ppa} GitHub repository.

\begin{abstract}
The course is a high-level introduction to program analysis with a
strong emphasis on its practical implementation in the design of programming languages
and code analyzers. Students may listen to this
course if they plan to develop their own programming languages,
compilers, IDEs, static and dynamic analyzers, code refactoring, generating and optimization tools.
The course combines theoretical study with the development of
instruments that analyze source code and automatically modifies it.
\end{abstract}

% \section*{Introduction}

\textbf{What is the goal?}\\
The primary objective of the course is to demonstrate
how theoretical knowledge of program analysis may be applied to
the design of software tools.

\textbf{Who is the teacher?}\\
Yegor is developing software for more than 30 years, being a hands-on programmer
(see his GitHub account: \href{https://github.com/yegor256}{@yegor256})
and a manager of other programmers. At the moment Yegor is a director
of an R\&D laboratory in Huawei. His primary research focus is
software quality problems. Some of the lectures he has recently presented
at some software conferences could be found at
\href{https://www.youtube.com/channel/UCr9qCdqXLm2SU0BIs6d_68Q}{his YouTube channel}.
Yegor also published a \href{https://www.yegor256.com/books.html}{few books}
and wrote a \href{https://www.yegor256.com/contents.html}{blog} about software engineering
and object-oriented programming.
Yegor previously tought two courses in
Innopolis University (Kazan, Russia)
and HSE University (Moscow, Russia):
\href{https://github.com/yegor256/ssd16}{Software Systems Design}
and
\href{https://github.com/yegor256/eqsp}{Ensuring Quality in Software Projects}
(all videos are available).

\textbf{Why this course?}\\
The quality of software code that most of us programmers write is way below
the expectations of our customers. Two main reasons for that
is \begin{inparaenum}[a)]
\item the lack of understanding of how programming languages are designed
internally and
\item the absences of connection between theoretical knowledge
about language design and the actual software we use every day to
write code: IDEs, compilers, code analyzers and modifiers.\end{inparaenum}
This course may help build the bridge between theory and practice.

\textbf{What's the methodology?}\\
The course is organized in pairs of lectures (45 minutes each). The first lecture in a pair
is an introduction of a theory, while the second lecture is a demonstration
of how the theory may be applied to the development of a software tool.
Either existing GitHub projects will be used for the demonstration
or new projects will be developed on-stage.

\newpage
\section*{Course Structure}

Prerequisites to the course (it is expected that a student knows this):

\begin{itemize}
\item How to write code
\item How to design software
\end{itemize}

After the course a student \emph{hopefully} will understand the basics of:

\begin{itemize}
\item Formal Grammar
\item Syntax Analysis
\item Abstract Syntax Tree
\item Formal Semantics
\item Abstract Machines
\item Program Analysis
\item Data Flow Analysis
\item Symbolic Execution
\item Model Checking
% %
% \item Type Theory
% \item $\lambda$-calculus
% \item Temporal Logic
\end{itemize}

Also, a student will be able to develop:

\begin{itemize}
\item A Programming Language
\item A Compiler
\item A Static Analyzer
\item A Code Refactoring Tool
\end{itemize}

\newpage
\section*{Lectures}

The following topics are discussed:

\newlist{lectures}{enumerate}{10}
\setlist[lectures]{label*=\arabic*.}
\begin{lectures}
\item Formal Grammar
	\begin{itemize}
	\item Notation
	\item Production Rules
	\item Parse Tree
	\item Ambiguity of Grammar
	\item Chomsky's Four Types of Grammars
	\item Regular Grammar
	\item Context Free Grammar (CFG)
	\item Linear Grammars
	\item Precedence and Associativity
	\item Recursive Rules
	\item Leftmost and Rightmost Derivation
	\item Non-deterministic CFG
	\item Left Factoring
	\end{itemize}
\item Syntax Analysis
	\begin{itemize}
	\item Extended Backus-Naur Form
	\item Lexical Analysis (Grammar + Lexer)
	\item Tokenization
	\item Syntactic Analysis (Parsing)
	\item Top-down (LL) and Bottom-up (LR) Parsing
	\item Flex and Bison
	\item ANTLR
	\item Off-side Rule
	\end{itemize}
\item Abstract Syntax Tree
	\begin{itemize}
	\item Contextual Analysis
	\item Semantic Analysis
	\item Intermediate Language
	\item Control Flow Graph
	\item AST in XML
	\end{itemize}
\item Formal Semantics
	\begin{itemize}
	\item Inference Rules, Axioms, Proof Trees
	\item Natural Semantics (Denotational)
	\item Structural Semantics (Operational)
	\item Reduction Semantics
	\end{itemize}
\item Abstract Machines
	\begin{itemize}
	\item Turing Machine
	\item Finite-State Machine
	\item SECD Machine
	\item Graph-based VM
	\end{itemize}
\item Program Analysis
	\begin{itemize}
	\item Rice's Theorem
	\item Static vs Dynamic Analysis
	\item Soundness and Completeness
	\item Precision and Recall
	\item Abstract Interpretation
	\item Approximation
	\item Lattices
	\end{itemize}
\item Data Flow Analysis
	\begin{itemize}
	\item Basics Blocks, Transfer Function, and Join Operation
	\item Work List Approach
	\item Forward and Backward Analysis
	\item Live Variable Analysis
	\item Definite Assignment Analysis
	\item Available Expressions Analysis
	\end{itemize}
\item Symbolic Execution
	\begin{itemize}
	\item Constraint Solvers (SAT/SMT)
	\item Itra- vs Inter-procedural Analysis
	\item Concolic Execution
	\item Path Explosion
	\item KLEE
	\end{itemize}
\item Model Checking
	\begin{itemize}
	\item Verification vs. Validation
	\item Program Graph
	\item Transformation System
	\item SPIN
	\end{itemize}
\end{lectures}

\newpage
\section*{Grading}

Students may form groups of up to three people. Each group will present
their own public GitHub repository with a software module inside, which
may be one of the following:

\begin{itemize}
	\item Compiler,
	\item Static analyzer,
	\item Transpiler,
	\item Code refactoring tool.
\end{itemize}

Higher grades will be given for (in this order):

\begin{itemize}
	\item Higher formalism of documentation,
	\item Higher complexity,
	and
	\item Higher test coverage,
\end{itemize}

Attendance will be tracked at the lectures.
If a student attends more than 75\% of all lectures,
they will not get less than ``C''.

At the laboratory classes each group will have to complete three
home works and defend them verbally on-site.
A completion of less than two will give everybody in the group a negative point,
a completion of three --- will give a positive point; the point will be added
to the grade given by the lecturer.

A retake exam is possible, following exactly the same procedure.
However, the highest mark possible at the retake is ``C.''

\newpage
\section*{Learning Material}

The following books are highly recommended to read (in no particular order):

\begin{multicols}{2}\small\raggedright
{Robert Harper}, \emph{Practical Foundations for Programming Languages}\\[3pt]
{Xavier Rival et al.}, \emph{Introduction to Static Analysis: An Abstract Interpretation Perspective}\\[3pt]
{Peter Linz}, \emph{An Introduction to Formal Languages and Automata}\\[3pt]
{Terence Parr}, \emph{The Definitive ANTLR 4 Reference}\\[3pt]
{Benjamin C. Pierce}, \emph{Programming Language Foundations}\\[3pt]
{Glynn Winskel}, \emph{Formal Semantics of Programming Languages}\\[3pt]
{Flemming Nielson et al.}, \emph{Principles of Program Analysis}\\[3pt]
{Anders Møller et al.}, \emph{Static Program Analysis}\\[3pt]
{Patrick Cousot}, \emph{Principles of Abstract Interpretation}\\[3pt]
{Uday Khedker et al.}, \emph{Data Flow Analysis: Theory and Practice}\\[3pt]
{Christel Baier}, \emph{Principles of Model Checking}
\end{multicols}

It is also recommended to watch YouTube lectures of
	\href{https://www.youtube.com/@MichaelPradel}{Michael Pradel}.
	\href{https://www.youtube.com/@nesoacademy}{@nesoacademy},
	\href{https://www.youtube.com/@DmitrySoshnikov-education}{Dmitry Soshnikov},
	\href{https://www.youtube.com/@MichaelRyanClarkson}{Michael Ryan Clarkson},
	\href{https://www.youtube.com/@krismicinski}{Kristopher Micinski},
	and
	\href{https://www.youtube.com/@htz4523}{Joost-Pieter Katoen}.

Also, check the \href{http://www.cs.cmu.edu/~aldrich/courses/17-355-18sp/}{Program Analysis}
course by Jonathan Aldrich and Claire Le Goues.

\end{document}
