% (The MIT License)
%
% Copyright (c) 2022-2023 Yegor Bugayenko
%
% Permission is hereby granted, free of charge, to any person obtaining a copy
% of this software and associated documentation files (the 'Software'), to deal
% in the Software without restriction, including without limitation the rights
% to use, copy, modify, merge, publish, distribute, sublicense, and/or sell
% copies of the Software, and to permit persons to whom the Software is
% furnished to do so, subject to the following conditions:
%
% The above copyright notice and this permission notice shall be included in all
% copies or substantial portions of the Software.
%
% THE SOFTWARE IS PROVIDED 'AS IS', WITHOUT WARRANTY OF ANY KIND, EXPRESS OR
% IMPLIED, INCLUDING BUT NOT LIMITED TO THE WARRANTIES OF MERCHANTABILITY,
% FITNESS FOR A PARTICULAR PURPOSE AND NONINFRINGEMENT. IN NO EVENT SHALL THE
% AUTHORS OR COPYRIGHT HOLDERS BE LIABLE FOR ANY CLAIM, DAMAGES OR OTHER
% LIABILITY, WHETHER IN AN ACTION OF CONTRACT, TORT OR OTHERWISE, ARISING FROM,
% OUT OF OR IN CONNECTION WITH THE SOFTWARE OR THE USE OR OTHER DEALINGS IN THE
% SOFTWARE.

\documentclass{article}
\usepackage{../ppa}
\newcommand*\thetitle{Program Analysis}
\newcommand*\thesubtitle{...}
\begin{document}

\plush{\innoTitlePage{6}}

\pptToc

\plush{\pptChapter{Basics}}

\pptSection[Property]{Syntactic \& Semantic Property}

\emph{Semantic} property can be completely defined with respect to the set of executions of a program,
while a \emph{syntactic} property can be decided directly based on the program text.

\plush{}

\pptSection{Rice's Theorem}

\emph{Rice's theorem} states that all non-trivial semantic properties of programs are undecidable.

A property is \emph{non-trivial} if it is neither true for every partial computable function, nor false for every partial computable function.

\emph{Halting problem} is the problem of determining, from 1)~a description of an arbitrary computer program and 2)~an input, whether the program will finish running, or continue to run forever. A general algorithm to solve the halting problem for all possible program–input pairs \textbf{cannot exist}.

\plush{}

\pptSection{Non-trivial Properties}

Examples of a non-trivial properties:

\begin{itemize}
\item A program exits
\item A program prints ``Hello''
\item A program dies with ``Segmentation Fault''
\item A program prints user password to the console
\end{itemize}

\plush{}

\pptSection{Static Analysis}

Consider two C++ programs given to a \emph{static analyzer} (e.g. \href{https://clang.llvm.org/extra/clang-tidy/}{Clang Tidy}):

\begin{multicols}{2}
{\ttfamily
int f() \{ \\
\quad int x = 0; \\
\quad return 42 / x; \\
\}}
\par\columnbreak\par
{\ttfamily
int f(int x) \{ \\
\quad return 42 / x; \\
\}}
\end{multicols}

Expected answers from Clang Tidy:

\begin{multicols}{2}
\textcolor{green}{\textbf{Yes! :)}}
\par\columnbreak\par
\textcolor{red}{\textbf{No :(}}
\end{multicols}

\plush{}

\pptSection{Style Checking}

Consider two C++ programs given to a \emph{style checker} (e.g. \href{https://github.com/cpplint/cpplint}{cpplint}):

\begin{multicols}{2}
{\ttfamily
int f (int x) \\
\{ \\
\quad return 42 / x; \\
\}}
\par\columnbreak\par
{\ttfamily
int f(int x) \{ \\
\quad return 42 / x; \\
\}}
\end{multicols}

Expected answers from cpplint:

\begin{multicols}{2}
\textcolor{green}{{\ttfamily Extra space before ( in function call ; \{ should almost always be at the end of the previous line}}
\par\columnbreak\par
\textcolor{red}{\textbf{No :(}}
\end{multicols}

\plush{}

\pptSection{Dynamic Analysis}

Consider this C++ programs given to a \emph{dynamic analyzer}:

\begin{multicols}{2}
{\ttfamily
int* foo() \{ \\
\quad return (int*) 42; \\
\} \\
int main() \{ \\
\quad int* p = foo(); \\
\quad return *p; \\
\}}
\par\columnbreak\par
{\ttfamily
\$ gcc main.cpp \\
\$ ./a.out \\
Segmentation fault: 11}
\end{multicols}

\plush{}

\plush{\pptChapter[Quality]{Quality of Analysis}}

\pptSection[Sound]{Sound \& Complete}

\pptPic{0.8}{sound-complete.jpg}

\plush{}

\pptSection[Metrics]{Precision \& Recall}

\emph{Precision} is the fraction of relevant instances among the retrieved instances (100\% precision means \emph{soundness}).

\emph{Recall} is the fraction of relevant instances that were retrieved (100\% recall means \emph{completness}).

\begin{pptWideOne}
\begin{equation*}
\text{Precision} = \frac{TP}{TP + FP}
\quad\quad
\text{Recall} = \frac{TP}{TP + FN}
\quad\quad
\text{Accuracy} = \frac{TP + TN}{TP + TN + FP + FN}
\end{equation*}
\end{pptWideOne}

\plush{}

\pptSection{Experiment}

Say, we give a few programs to a static analyzer:

\begin{tikzpicture}[node distance=3cm,
box/.style={rectangle,draw,minimum height=2cm,minimum width=2cm,line width=2pt},
bad/.style={box,color=red},
good/.style={box,color=green},
report/.style={},
missed/.style={report,color=red},
found/.style={report,color=green}]
\node[good] (a) {a} node[found,below=.5cm of a] {Yes};
\node[good,right of=a] (b) {b} node[found,below=.5cm of b] {Yes};
\node[good,right of=b] (c) {c} node[missed,below=.5cm of c] {No};
\node[good,right of=c] (d) {d} node[found,below=.5cm of d] {Yes};
\node[bad,right of=d] (e) {e} node[missed,below=.5cm of e] {No};
\node[bad,right of=e] (f) {f} node[found,below=.5cm of f] {Yes};
\node[bad,right of=f] (g) {g} node[found,below=.5cm of g] {Yes};
\node[bad,right of=g] (h) {h} node[missed,below=.5cm of h] {No};
\end{tikzpicture}

\begin{pptWideOne}
\begin{gather*}
TP = \underline{\hspace{1cm}}
\quad\quad
FP = \underline{\hspace{1cm}}
\quad\quad
TN = \underline{\hspace{1cm}}
\quad\quad
FN = \underline{\hspace{1cm}}
\\
\text{Precision} = \frac{TP}{TP + FP} = \underline{\hspace{5cm}}
\quad\quad
\text{Recall} = \frac{TP}{TP + FN} = \underline{\hspace{5cm}}
\\
\text{Accuracy} = \frac{TP + TN}{TP + TN + FP + FN} = \underline{\hspace{5cm}}
\\
\end{gather*}
\end{pptWideOne}

\plush{}

\plush{\pptChapter{Abstract Interpretation}}

\plush{\pptChapter{Approximation}}

\end{document}
